\documentclass[12pt]{article}
\usepackage[utf8]{inputenc}
\usepackage[spanish]{babel}
\usepackage{amsmath}
\usepackage{amsfonts}
\usepackage{amssymb}
\usepackage{amsthm}
\usepackage{blindtext}
\usepackage{mathtools}
\usepackage{graphicx}
\usepackage{latexsym}
\usepackage{cancel}
\usepackage[left=2cm,top=2cm,right=2cm,bottom=2cm]{geometry}
\usepackage[all]{xy}
\usepackage{cancel}
\usepackage{pictexwd}
\usepackage{parskip}
%\usepackage{pgfplots}    ESTOS DOS HAY QUE COMENTARLOS (A VECES NO HACE FALTA)
%\pgfplotsset{compat=1.15}
\usepackage{mathrsfs}
\usepackage{vmargin}
\usepackage{hyperref}

\DeclarePairedDelimiter\Floor\lfloor\rfloor
\DeclarePairedDelimiter\Ceil\lceil\rceil


\newtheorem{theorem}{Teorema}[section]
\newtheorem{definicion}[theorem]{Definición}
\newtheorem{proposition}[theorem]{Proposición}
\newtheorem{lemma}{Lema}[theorem]
\newtheorem{definition}[theorem]{Definición}
\newtheorem{example}{Ejemplo}[theorem]
\newtheorem{corolario}{Corolario}[theorem]
\newtheorem{observation}{Observación}[theorem]
\newtheorem{properties}{Propiedades}[theorem]
\providecommand{\abs}[1]{\lvert#1\rvert}
\providecommand{\norm}[1]{\lVert#1\rVert}


\author{Pablo Pallàs}
\title{Teoría de funciones de una variable real}
\setlength{\parindent}{10pt}


\begin{document}
\rmfamily
\maketitle
\tableofcontents
\parindent= 0cm


\section{Introducción}
\section{Conceptos previos}
\section{Sucesiones}
\section{Continuidad}
\subsection{Límites de funciones}

\begin{definition}Dado un $a \in \mathbb{R}$, un conjunto $V \subseteq \mathbb{R}$ es un \textbf{entorno} de $a$ si contiene un intervalo de la forma $(a - \varepsilon, a + \varepsilon)$ para algún $\varepsilon >0$. Si $V$ es un entorno de $a$, diremos que $V \setminus \lbrace a \rbrace$ es un \textbf{entorno reducido} de $a$.
\end{definition}

Notar que todo conjunto que contenga un entorno de un punto también será a su vez entorno de ese punto.

\begin{definition}Sea $D \subseteq \mathbb{R}$ y $a \in \mathbb{R}$, entonces diremos que $a$ es un \textbf{punto de acumulación} de $D$ si todo entorno reducido de $a$ contiene puntos de $D$. Dicho de otra forma, si para cada $\varepsilon >0$ existe algún $y \in D$ tal que $y \neq a$, con $|y-a| < \varepsilon$, es decir, $0 < |y-a| < \varepsilon$.
\end{definition}

\begin{definition}El conjunto de puntos de acumulación de un conjunto $D$ suele denominarse \textbf{conjunto derivado} y se denota por $D'$.
\end{definition}

Así, podemos decir de forma intuitiva que $a \in D'$ si y sólo si hay puntos de $D$, distintos de $a$, arbitrariamente próximos al punto $a$.

\begin{example}Veamos algunos ejemplos: 
\begin{enumerate}
\item Si $D$ es finito, entonces $D' = \emptyset$.
\item $\mathbb{N}' = \mathbb{Z}' = \emptyset$, y $\mathbb{Q}' = \mathbb{R}$.
\item $(a,b)' = [a,b]' = [a,b]$.
\item Si $D = \lbrace 1/n : n \in \mathbb{N} \rbrace$, entonces $0 \in D'$ a pesar de que $0 \notin D$ y $1 \notin D'$ a pesar de que $1 \in D$.
\end{enumerate}
\end{example}

Podemos probar que $a \in D'$ si y sólo si existe una sucesión $(x_n)$ de puntos de $D$ distintos de $a$ que converge al punto $a$.

\begin{definition}[\textbf{\textit{Límite de una función en un punto}}]Sea $D \subseteq \mathbb{R}$, $f \colon D \longrightarrow \mathbb{R}$, $a \in D'$, $b \in \mathbb{R}$. Escribiremos $$\lim_{x\rightarrow a} f(x) = b$$ cuando se cumpla que para cada $\varepsilon >0$ existe algún $\delta >0$ tal que $\forall x \in D$ con $0 < |x-a | < \delta$ se tiene $|f(x) -b| < \varepsilon$.

Entonces diremos que $b$ es el \textbf{límite} de $f(x)$ cuando $x$ tiende al punto $a$.
\end{definition}

La condición de que $|f(x) -b | < \varepsilon$ para todo $x \in D$ con $0 < |x-a| < \delta$ se puede escribir de otra forma: $$f(U) \subseteq (b- \varepsilon, b + \varepsilon), \quad U = [D \cap (a- \delta, a + \delta)] \setminus \lbrace a \rbrace.$$

Podemos decir de forma resumida que $b$ será el límite de $f(x)$ cuando $x$ tiende a $a$ si $f(x)$ se acerca a $b$ cuando $x$ se acerca al punto $a$, aunque sin tomar su valor, dentro del dominio de $f$. Esto último es muy importante.

\begin{proposition}[\textbf{\textit{Unicidad del límite}}] Sea $D \subseteq \mathbb{R}$, $f \colon D \longrightarrow \mathbb{R}$, $a \in D'$, $b_1,b_2 \in \mathbb{R}$. Si $$\lim_{x \rightarrow a} f(x) = b_1, \quad \lim_{x\rightarrow a} f(x) = b_2,$$ entonces $b_1 = b_2.$
\end{proposition}
\begin{proof}
Supongamos, por ejemplo, que $b_1 < b_2$. Elijamos $\varepsilon = \dfrac{b_2-b_1}{2}$. Deben existir entonces $\delta_1 >0$ tal que para todo $x \in D$ con $0 <|x-a| < \delta_1$ tenemos $f(x) < b_1 + \varepsilon = \dfrac{b_1+b_2}{2}$ y un $\delta_2 >0$ tal que para todo $x \in D$ con $0 < |x-a | < \delta_2$ se tiene que $\dfrac{b_1+b_2}{2} = b_2-\varepsilon < f(x)$. Definiendo $\delta = \min{\delta_1, \delta_2}$, resulta que para todo $x \in D$ con $0 < |x-a | < \delta$ tenemos $\dfrac{b_1+b_2}{2} < f(x) < \dfrac{b_1+b_2}{2}$, esto es absurdo. Análogo si $b_2 <b_1$.

\end{proof}

\begin{proposition}\label{eq:limsuc} Sea $D \subseteq \mathbb{R}$, $f \colon D \longrightarrow \mathbb{R}$, $a \in D'$, $b \in \mathbb{R}$. Son equivalentes: 
\begin{enumerate}
\item $\lim_{x\rightarrow a} f(x) = b$.
\item Para cada sucesión $(s_n)$ de puntos de $D \setminus \lbrace a \rbrace $ tal que $\lim_n s_n = a$ se verifica $\lim_nf(s_n) = b$.
\end{enumerate}
\end{proposition}
\begin{proof}
Supongamos que $\lim_{x \rightarrow a} f(x) = b$. Para cualquier $\varepsilon >0$ se puede encontrar un $\delta >0$ de modo que para todo $x \in D$ con $0 < |x-a|< \delta$ se cumple $|f(x) -b | < \varepsilon$. Sea $(s_n)$ una sucesión de puntos de $D \setminus \lbrace a \rbrace$ tal que $\lim_n s_n = a$. Dado $\delta >0$, existirá un $N \in \mathbb{N}$ tal que para todo $n > N$ se verifica $|s_n-a | < \delta$, y como $s_n \neq a$, deducimos que $|f(s_n) -b | < \varepsilon$, es decir, $\lim_n f(s_n) = b$.

Recíprocamente, probaremos que si $1.$ no se cumple entonces $2.$ tampoco. Que no se cumpla $1.$ quiere decir que existe algún $\varepsilon >0$ tal que para todo $\delta >0$ hay al menos un $x_\delta \in D$ que cumple $0 < |x_\delta - a | < \delta$ y, sin embargo, $|f(x_\delta)-b | \geq \varepsilon$.
Para cada $n \in \mathbb{N}$, elegimos $\delta = 1/n$. Hay algún punto $s_n \in D$ que cumple $0 < |s_n -a| < 1/n$ y, sin embargo, $|f(s_n)-b | \geq \varepsilon$. La sucesión $(s_n)$ así obtenida tiene las siguientes propiedades: 
\begin{enumerate}
\item Está contenida en $D \setminus \lbrace a \rbrace$, porque $s_n \in D$, pero $0< |s_n -a|$. 
\item $\lim_n s_n=a$ porque $0 < |s_n-a|<1/n$ (basta aplicar la definición de límite).
\item La sucesión $f(s_n)$ no tiende a $b$ porque para todos los $n \in \mathbb{N}$, $|f(s_n)-b| \geq \varepsilon$.
\end{enumerate}
Por lo tanto, no se cumple $2.$
\end{proof}

Igualmente, diremos que un conjunto $V \subseteq \mathbb{R}$ es un \textbf{entorno reducido} de $+ \infty$ ó de $-\infty$ si existe un $r \in \mathbb{R}$ tal que $(r, + \infty) \subseteq V$ ó respectivamente $(- \infty, r) \subseteq V$.

\begin{definition} Se dice que $+ \infty	$ es un punto de acumulación de un conjunto $D \subseteq \mathbb{R}$ si $D$ no está acotado superiormente, y escribiremos $+ \infty \in D'$. Igualmente, diremos que $- \infty$ es un punto de acumulación de un conjunto $D \subseteq \mathbb{R}$ si $D$ no está acotado inferiormente, y escribiremos $- \infty	\in D'$.
\end{definition}

\begin{definition} Sea $D \subseteq \mathbb{R}$, $f \colon D \longrightarrow \mathbb{R}$, $a,b \in \mathbb{R} \cup \lbrace \pm \infty \rbrace$, $a \in D'$. Escribiremos $$\lim_{x \rightarrow a } f(x) = b$$ si para cada entorno $V$ de $b$ existe un entorno reducido $U$ de $a$ tal que $f(U) \subseteq V$.
\end{definition}

Pueden darse definiciones en términos de desigualdades, desglosando los diferentes casos posibles. Concretamente, sean $D \subseteq \mathbb{R}$, $f \colon D \longrightarrow \mathbb{R}$, $a,b \in \mathbb{R}$. Entonces
\begin{enumerate}
\item $\lim_{x \rightarrow a} f(x) = + \infty	$ si para cada $M \in \mathbb{R}$ existe algún $\delta >0$ tal que todos los $x \in D$ con $0 < |x-a | < \delta$ cumplen $f(x) > M$.
\item $\lim_{x \rightarrow a} f(x) = - \infty	$ si para cada $M \in \mathbb{R}$ existe algún $\delta >0$ tal que todos los $x \in D$ con $0 < |x-a | < \delta$ cumplen $f(x) < M$.
\item $\lim_{x \rightarrow + \infty} f(x) = b	$ si para cada $\varepsilon >0 $ existe algún $K \in \mathbb{R}$ tal que todos los $x \in D$ con $x >K$ cumplen $|f(x) -b | < \varepsilon$.
\item $\lim_{x \rightarrow + \infty} f(x) = + \infty	$ si para cada $M \in \mathbb{R}$ existe algún $K \in \mathbb{R}$ tal que todos los $x \in D$ con $x>K$ cumplen $f(x) >M$. 
\item $\lim_{x \rightarrow + \infty} f(x) = - \infty	$ si para cada $M \in \mathbb{R}$ existe algún $K \in \mathbb{R}$ tal que todos los $x \in D$ con $x>K$ cumplen $f(x) <M$. 
\item $\lim_{x \rightarrow -\infty} f(x) = b	$ si para cada $\varepsilon >0 $ existe algún $K \in \mathbb{R}$ tal que todos los $x \in D$ con $x <K$ cumplen $|f(x) -b | < \varepsilon$.
\item $\lim_{x \rightarrow - \infty} f(x) = + \infty	$ si para cada $M \in \mathbb{R}$ existe algún $K \in \mathbb{R}$ tal que todos los $x \in D$ con $x<K$ cumplen $f(x) >M$.
\item $\lim_{x \rightarrow - \infty} f(x) = - \infty	$ si para cada $M \in \mathbb{R}$ existe algún $K \in \mathbb{R}$ tal que todos los $x \in D$ con $x<K$ cumplen $f(x) >M$.
\end{enumerate}

Con todo esto, sigue habiendo unicidad de límite e igualmente se mantiene la caracterización mediante sucesiones: 
\begin{proposition}\label{eq:maslimsuc} Sea $D \subseteq \mathbb{R}$, $f \colon D \longrightarrow \mathbb{R}$, $a,b \in \mathbb{R} \cup \lbrace \pm \infty \rbrace$, $a \in D'$. Son equivalentes: 
\begin{enumerate}
\item $\lim_{x \rightarrow a} f(x) = b$.
\item Para cada sucesión $(s_n)$ de puntos de $D \setminus \lbrace a \rbrace$ tal que $\lim_n s_n = a$ se verifica $\lim_n f(s_n) = b$.
\end{enumerate}
\end{proposition}
\begin{proof}
Basta adaptar a cada caso la demostración de~\ref{eq:limsuc}.
\end{proof}

\begin{proposition}[\textbf{\textit{Operaciones algebraicas con límites}}] Sean $D \subseteq \mathbb{R}$, $a \in \mathbb{R} \cup \lbrace \pm \infty \rbrace$ un punto de acumulación de $D$, $c \in \mathbb{R}$ y $f,g \colon D \longrightarrow \mathbb{R}$. Se tiene entonces: 
\begin{enumerate}
\item $\lim_{x \rightarrow a } (f(x) + g(x)) = \lim_{x \rightarrow a} f(x) + \lim_{x \rightarrow a} g(x)$, si estos últimos límites existen y su suma está definida en $\mathbb{R} \cup \lbrace \pm \infty \rbrace$.
\item $\lim_{x \rightarrow a} cf(x) = c \lim_{x \rightarrow a} f(x)$, si este último límite existe y su producto por $c$ está definido en $\mathbb{R} \cup \lbrace \pm \infty	 \rbrace$.
\item $\lim_{x \rightarrow a} f(x)g(x) = \lim_{x \rightarrow a} f(x) \lim_{x \rightarrow a} g(x)$, si estos últimos existen y su producto está definido en $\mathbb{R} \cup \lbrace \pm \infty	\rbrace$. 
\item $\lim_{x \rightarrow a} \dfrac{f(x)}{g(x)} = \dfrac{\lim_{x \rightarrow a} f(x)}{\lim_{x \rightarrow a}g(x)}$, si estos últimos límites existen y su cociente está definido en $\mathbb{R} \cup \lbrace \pm \infty \rbrace.$
\end{enumerate}
\end{proposition}
\begin{proof}
Basta aplicar el resultado anterior y el análogo para sucesiones.

\end{proof}

\begin{proposition}[\textbf{\textit{Acotación y límite cero}}]
Sean $D \subseteq \mathbb{R}$, $a \in \mathbb{R} \cup \lbrace \pm \infty	 \rbrace$ un punto de acumulación de $D$, y $f, g \colon D \longrightarrow \mathbb{R}$. Supongamos que 
\begin{enumerate}
\item La función $f$ está acotada, es decir, existe $M >0$ tal que $|f(x)| \leq M$ para todo $x \in D$.
\item $\lim_{x  \rightarrow a} g(x) = 0$. 
\end{enumerate}
Entonces, $\lim_{x \rightarrow a} f(x)g(x) = 0$.
\end{proposition}
\begin{proof}
Igual que antes, aplicar~\ref{eq:maslimsuc} y el resultado análogo para sucesiones.

\end{proof}

\begin{proposition}[\textbf{\textit{Cambios de variable}}]Sean $D, E$ subconjuntos de $\mathbb{R}$, $a$ un punto de acumulación de $D$, $b$ un punto de acumulación de $E$, $f \colon D \longrightarrow \mathbb{R}$, y $g \colon E \longrightarrow \mathbb{R}$ tales que $f(D) \subseteq E$ y supongamos que $$\lim_{x \rightarrow a} f(x) = b, \lim_{y \rightarrow b} g(y) = c.$$ Si $b \notin f(D)$, entonces existe $\lim_{x \rightarrow a} g(f(x)) = c$.
\end{proposition}
\begin{proof}
Sea $\varepsilon >0$. Como $\lim_{y \rightarrow b } g(y) = c$, existe algún $r >0$ tal que para todo $y \in E$ con $0 < |y-b |<r$, se tiene que $|g(y) -c | < \varepsilon$. 

Ahora, como $\lim_{x \rightarrow a} f(x) = b$, existe algún $\delta >0$ tal que para todo $x \in D$ con $0 < |x-a| < \delta$, se tiene $ |f(x) - b | < r$.

Sea $x \in D$, con $0 < |x-a| < \delta$. No sólo es $|f(x)-b| < r$, sino que como $b  \notin f(D)$ y $f(D) \subseteq E$, resulta $$0 < |f(x) -b | < r, \quad f(x) \in E.$$ Por lo tanto, $|g(f(x)) -c| < \varepsilon$.

\end{proof}

A veces es útil en el cálculo de límites tener en cuenta las siguientes consecuencias inmediatas de la definición de límite: 

\begin{proposition}Si $D \subseteq \mathbb{R}$, $a$ es un punto de acumulación de $D$ y $f \colon D \longrightarrow \mathbb{R}$, 
\begin{enumerate}
\item $\lim_{x\rightarrow a} f(x) = b \in \mathbb{R} \Longleftrightarrow \lim_{x \rightarrow a} |f(x) -b | = 0$.
\item $\lim_{x \rightarrow a} f(x) = b \in \mathbb{R} \Longrightarrow \lim_{x \rightarrow a} |f(x) | = |b|$. El recíproco, en general, sólo es cierto cuando $b= 0$.
\item $\lim_{x \rightarrow a} f(x) = b \Longleftrightarrow \lim_{t \rightarrow 0} f(a+t) = b$.
\end{enumerate}
\end{proposition}

Si en las definiciones de límites añadimos una de las dos condiciones, $x>a$, $x <a$, entonces se habla de límites laterales (por la derecha y por la izquierda). Emplearemos la notación $\lim_{x \rightarrow a^+} f(x), \lim_{x \rightarrow a^-} f(x)$.

\begin{definition}[\textbf{\textit{Límites laterales: por la derecha y por la izquierda}}]
Sean $D \subseteq \mathbb{R}$, $f \colon D \longrightarrow \mathbb{R}$, $a \in \mathbb{R}$ un punto de acumulación de $D$ y $b \in \mathbb{R}$.
\begin{enumerate}
\item Se dice que $\lim_{x \rightarrow a^+}f(x) = b$ si para cada $\varepsilon >0$ existe algún $\delta >0$ tal que todos los $x \in D$ con $0< x-a < \delta$ cumplen $|f(x) -b | < \varepsilon$.
\item Se dice que $\lim_{x \rightarrow a^+}f(x) = +\infty$ si para cada $M \in \mathbb{R}$ existe algún $\delta >0$ tal que todos los $x \in D$ con $0< x-a < \delta$ cumplen $f(x) >M$.
\item Se dice que $\lim_{x \rightarrow a^+}f(x) = -\infty$ si para cada $M \in \mathbb{R}$ existe algún $\delta >0$ tal que todos los $x \in D$ con $0< x-a < \delta$ cumplen $f(x) <M$.
\item Se dice que $\lim_{x \rightarrow a^-}f(x) = b$ si para cada $\varepsilon >0$ existe algún $\delta >0$ tal que todos los $x \in D$ con $0< a-x < \delta$ cumplen $|f(x) -b | < \varepsilon$.
\item Se dice que $\lim_{x \rightarrow a^-}f(x) = +\infty$ si para cada $M \in \mathbb{R}$ existe algún $\delta >0$ tal que todos los $x \in D$ con $0< a-x < \delta$ cumplen $f(x) >M$.
\item Se dice que $\lim_{x \rightarrow a^-}f(x) = -\infty$ si para cada $M \in \mathbb{R}$ existe algún $\delta >0$ tal que todos los $x \in D$ con $0< a-x < \delta$ cumplen $f(x) <M$.
\end{enumerate}
\end{definition}

Como consecuencia inmediata de las definiciones tenemos: 

\begin{proposition}Sean $D \subseteq \mathbb{R}$, $f \colon D \longrightarrow \mathbb{R}$ y $a \in \mathbb{R}$ de modo que $(a - \delta, a+ \delta) \subseteq D$ para algún $\delta >0$. Sea $b \in \mathbb{R} \cup \lbrace \pm \infty \rbrace$. Entonces, $$\lim_{x \rightarrow a}f(x) = b \Longleftrightarrow \lim_{x\rightarrow a^-}f(x) = \lim_{x \rightarrow a^+} f(x) = b.$$
\end{proposition}

\begin{proposition}Sean $D \subseteq \mathbb{R}$, $f \colon D \longrightarrow \mathbb{R}$ monótona no decreciente, $a \in \mathbb{R} \cup \lbrace \pm \infty \rbrace$. 
\begin{enumerate}
\item Si $a \in [D \cap (- \infty, a)]'$, entonces $f$ tiene \textbf{límite por la izquierda} en $a$ (finito o infinito) y es $$\lim_{x \rightarrow a^-} f(x) = \sup \lbrace f(x) : x \in D \cap (-\infty,a) \rbrace$$ (entendindo que, si el conjunto no está acotado superiormente, su supremo es $+ \infty$).
\item Si $a \in [D \cap (- \infty, a)]'$, entonces $f$ tiene \textbf{límite por la derecha} en $a$ (finito o infinito) y es $$\lim_{x \rightarrow a^+} f(x) = \inf \lbrace f(x) : x \in D \cap (a,+\infty) \rbrace$$ (entendindo que, si el conjunto no está acotado inferiormente, su ínfimo es $- \infty$).
\end{enumerate}
\end{proposition}

Pasemos ahora a estudiar algunos límites básicos de funciones elementales.

Si $f(x)$ representa una función cualquiera de las siguientes: $e^x$, $\log x$, $\sin x$, $\cos x$, $\tan x$, $\arcsin x$, $\arccos x$, $\arctan x$, $x^r$, entonces tendremos que $$\lim_{x\rightarrow a} f(x) = f(a)$$ par cualquier punto $a$ del dominio de la función. Algunos límites importantes son: \begin{enumerate}
\item $\lim_{x \rightarrow - \infty} e^x = 0.$
\item $\lim_{x \rightarrow +\infty} e^x = + \infty.$
\item $\lim_{x \rightarrow 0^+} \log x = - \infty.$
\item $\lim_{x \rightarrow +\infty} \log x = +\infty.$
\item $\lim_{x \rightarrow (\pi /2 )^-} \tan x = +\infty.$
\item $\lim_{x \rightarrow (\pi /2 )^+} \tan x = -\infty.$
\item $\lim_{x \rightarrow -\infty} \arctan x = -\dfrac{\pi}{2}.$
\item $\lim_{x \rightarrow +\infty} \arctan x = \dfrac{\pi}{2}.$
\item $\lim_{x \rightarrow 0^+} x^r = 0$, si $r>0.$
\item $\lim_{x \rightarrow +\infty} x^r = +\infty$, si $r>0$.
\item $\lim_{x \rightarrow 0^+} x^r = +\infty$, si $r<0$.
\item $\lim_{x \rightarrow +\infty} x^r = 0,$ si $r<0$.
\item Si $f(x) = a_rx^r+a_{r-1}x^{r-1} + \ldots + a_0$ es un polinomio, con $r \in \mathbb{N}$ y $a_r \neq 0$, entonces $$\lim_{x \rightarrow +\infty} f(x) = +\infty \quad~si~a_r>0 ,$$ $$\lim_{x \rightarrow +\infty} f(x) = -\infty \quad~si~a_r<0.$$
\end{enumerate}

También tener en cuenta el conocido como \textit{orden de infinitud}, con $a>0$ y $b>1$: $$\log x \ll x^a \ll b^x \ll x^x \quad x \rightarrow + \infty.$$ Donde $f(x) \ll g(x)$ cuando $x \rightarrow + \infty$ significa que $$\lim_{x \rightarrow + \infty} \dfrac{f(x)}{g(x)} = 0.$$

\begin{definition}Sean $D \subseteq \mathbb{R}$, $a \in \mathbb{R} \cup \lbrace \pm \infty \rbrace$ un punto de acumulación de $D$ y $f,g \colon D \rightarrow \mathbb{R}$. Diremos que $f$ es equivalente a $g$ cuando $x$ tiende al punto $a$, y escribiremos $$f(x) \sim g(x) \quad (x \rightarrow a)$$ si se cumple que $$\lim_{x \rightarrow x} \dfrac{f(x)}{g(x)} = 1.$$
\end{definition}

Se pueden trasladar las equivalencias clásicas de sucesiones a equivalencias entre funciones. Tenemos así: 
\begin{enumerate}
\item Equivalencias de infinitésimos, es decir, cuando $x$ se hace muy pequeño ($x \rightarrow 0$): 
$$e^x - 1 \sim x$$ $$\log (1+x) \sim x$$ $$\sin x \sim x$$ $$1 - \cos x \sim x^2/2$$ $$\tan x \sim x$$ $$(1+x)^\alpha - 1 \sim \alpha x$$ $$\arcsin x \sim x$$ $$\arctan x \sim x.$$
\item Equivalencias de infinitos, es decir, cuando $x$ se hace muy grande ($x \rightarrow +\infty$), sea $f(x) = a_rx^r + a_{r-1}x^{r-1} + \ldots + a_0$, con $a_r \neq 0$: $$f(x) \sim a_rx^r$$ $$\log f(x) \sim r\log x \quad~si~a_r>0.$$
\end{enumerate}






\end{document}