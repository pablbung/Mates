\documentclass[12pt]{article}
\usepackage[utf8]{inputenc}
\usepackage[spanish]{babel}
\usepackage{amsmath}
\usepackage{amsfonts}
\usepackage{amssymb}
\usepackage{amsthm}
\usepackage{blindtext}
\usepackage{mathtools}
\usepackage{graphicx}
\usepackage{latexsym}
\usepackage{cancel}
\usepackage[left=2cm,top=2cm,right=2cm,bottom=2cm]{geometry}
\usepackage[all]{xy}
\usepackage{cancel}
\usepackage{pictexwd}
\usepackage{parskip}
\usepackage{pgfplots}
\pgfplotsset{compat=1.15}
\usepackage{mathrsfs}
\usepackage{vmargin}
\usepackage{graphicx}
%activar hyperref para moverse rápidamente en el documento, desactivar para visualizar o imprimir
%\usepackage{hyperref} 

\DeclarePairedDelimiter\Floor\lfloor\rfloor
\DeclarePairedDelimiter\Ceil\lceil\rceil


\newtheorem{theorem}{Teorema}[section]
\newtheorem{definicion}[theorem]{Definición}
\newtheorem{proposition}[theorem]{Proposición}
\newtheorem{lemma}{Lema}[theorem]
\newtheorem{definition}[theorem]{Definición}
\newtheorem{example}{Ejemplo}[theorem]
\newtheorem{corolario}{Corolario}[theorem]
\newtheorem{observation}{Observación}[theorem]
\newtheorem{properties}{Propiedades}[theorem]
\newtheorem{exercise}{Ejercicio}
\providecommand{\abs}[1]{\lvert#1\rvert}
\providecommand{\norm}[1]{\lVert#1\rVert}
\graphicspath{{images/}}

\author{Pablo Pallàs}
\title{Introducción a la Teoría de Grupos (UNED)}
\setlength{\parindent}{10pt}
\newpage
\begin{document}
\rmfamily
\maketitle
\vspace{2.5cm}
\tableofcontents
\parindent= 0cm
\newpage

\section{Grupos. Subgrupos. Índice de un subgrupo}
\subsection{Generalidades. Grupos}
Empezaremos por el principio del todo, y como estamos en \textit{Teoría de grupos} qué mejor forma de empezar que por una buena y sencilla definición de lo que son los grupos, los principales protagonistas de este libro.

\begin{definition}Un \textbf{grupo} es un conjunto no vacío $G$ en el que está definida una operación binaria $$\begin{array}{rccl}
&G \times G & \longrightarrow & G\\
&(a,b) & \longmapsto &ab
\end{array}
$$
que satisface: \begin{enumerate}
\item $(ab)c= a(bc)$ para cada terna de elementos $a,b,c$ de $G$. Se dice que la operación es \textbf{asociativa}.
\item Existe un elemento $e \in G$ tal que $$ea = a = ae \quad \forall a \in G.$$
\item Para cada elemento $a \in G$ existe un $x \in G$ tal que $$ax = e = xa.$$
\end{enumerate}
Diremos que $ab$ es el \textbf{producto} de $a$ por $b$.
\end{definition}

Veamos algunas observaciones respecto a la definición:

\begin{observation}El elemento $e$ de la segunda condición de la definición anterior es 
único, ya que si existiese otro $e'$ que verificara esa condición tendríamos $$ee' = e' = e'e$$ $$e'e = e = ee'$$ y así $e' = e'e = e.$

Se dice que $e$ es el \textbf{elemento neutro de $G$}. Usualmente lo denotaremos por $1_G$, y si no hay posible confusión con el grupo en el que estemos trabajando simplemente escribiremos $1$.
\end{observation}

\begin{observation}Si la operación en $G$ la notamos por $(a,b)\longmapsto a+b$, entonces la denominaremos \textbf{suma} y al elemento neutro $0_G$ o simplemente $0$.
\end{observation}

\begin{observation}Para cada $a \in G$, el elemento $x$ de la tercera condición es único puesto que si $y \in G$ cumpliese también esa condición tendríamos: $$ax = e = xa,$$ $$ay = e = ya.$$ En particular $ax = ay$, luego $x(ax) = x(ay),$ y por la propiedad asociativa $(xa)x = (xa)y,$ esto es, $ex = ey$ y así $x=y$.

Al único elemento $x \in G$ que cumple $$ax = e = xa$$ le denominaremos \textbf{inverso} de $a$ y lo notamos por $a^{-1}$. Nótese que si $a \in G$, como $aa^{-1} = 1_G = a^{-1}a$, $a$ es el inverso de $a^{-1}$, es decir, $$(a^{-1}){-1} = a.$$

Por último, apuntar que cuando la operación en $G$ sea la suma escribiremos $-a$ en vez de $a^{-1}$ y se denominará \textbf{opuesto} de $a$.
\end{observation}

\begin{proposition}[\textbf{\textit{Simplificación}}] 
Sean $a,b, c \in G$. Entonces:
\begin{enumerate}
\item Si $ab = ac$, entonces $b = c$.
\item Si $ba = ca$, entonces $b = c$.
\end{enumerate}
\end{proposition}
\begin{proof}
Si $ab = ac$, se tiene $a^{-1}(ab) = a^{-1}(ac)$ y así $(a^{-1}a)b = (a^{-1}a)c$, esto es, $b = 1b = 1c = c$. Análogamente con $ba = ca$.

\end{proof}

\begin{proposition}[\textbf{\textit{Asociatividad generalizada}}] Los productos que se obtienen al variar las formas de asociar $n$ elementos $a_1, \ldots, a_n$ de un grupo $G$, \textit{conservando el orden}, son iguales. Denotaremos cualquiera de esos productos por $a_1 \ldots a_n$.
\end{proposition}
\begin{proof}
Probaremos esto por inducción sobre $n$. Los casos $n = 1,2$ son evidentes. Supongamos $n>2$. Debemos demostrar que, si $1 < k<l<n,$ $$(a_1 \ldots a_k)(a_{k+1}\ldots a_n) = (a_1 \ldots a_l)(a_{l+1} \ldots a_n).$$ Sean $a = a_1 \ldots a_k$, $b = a_{k+1} \ldots a_l, c = a_{l+1} \ldots a_n$. Por la hipótesis de inducción,  $$a_1\ldots a_l = (a_1 \ldots a_k)(a_{k+1} \ldots a_l) = ab,$$ $$a_{k+1} \ldots a_n = (a_{k+1}\ldots a_l)(a_{l+1} \ldots a_n) = bc.$$ Así, lo que inicialmente queríamos probar equivale a probar que $$a(bc) = (ab)c,$$ lo cual es cierto por la propiedad asociativa.

\end{proof}

En particular, esto nos permite dar la siguiente definición: 

\begin{definition}Dado un elemento $a\in G$, y un natural $n$, definimos la \textbf{potencia $n$-ésima} de $a$ $$a^n = a \underbrace{\ldots}_n a.$$ Y para completar la definición consideraremos $a^0 =1$, $a^{-n} = (a^{-1})^n$.

Además, la ley de asociatividad generalizada nos permite deducir, con $m,n \in \mathbb{Z}$ y $a\in G$, $$a^ma^n = a^{m+n},$$ $$(a^m)^n = a^{mn}.$$
\end{definition}

\begin{proposition}Dados elementos $a_1, \ldots, a_n$ en un grupo $G$ se tiene $$(a_1 \ldots a_n)^{-1} = a_n^{-1} \ldots a_1^{-1}.$$
\end{proposition}
\begin{proof}
Lo probaremos por inducción. Si $n = 1$, es evidente. Si $n>1$, usando la asociatividad generalizada \begin{center}$(a_1\ldots a_n)(a_n^{-1} \ldots a_1^{-1}) = (a_1\ldots a_{n-1})(a_na_n^{-1})(a_{n-1}^{-1} \ldots a_1^{-1}) = (a_1\ldots a_{n-1})(a_{n-1}^{-1} \ldots a_1^{-1}) = \ldots = 1 = \ldots = (a_n^{-1} \ldots a_2^{-1}) (a_2\ldots a_n) =(a_n^{-1} \ldots a_2^{-1})(a_1^{-1}a_1)(a_2\ldots a_n) = (a_n^{-1} \ldots a_1^{-1})(a_1\ldots a_n)$\end{center}

\end{proof}

\begin{example}Veamos algunos ejemplos:
\begin{enumerate}
\item Los conjuntos $\mathbb{Z}, \mathbb{Q}, \mathbb{R}$ y $\mathbb{C}$ con la suma usual son grupos cuyo neutro es el número cero.
\item Los conjuntos $\mathbb{Q}^\ast, \mathbb{R}^\ast, \mathbb{C}^\ast$ obtenidos a partir de $\mathbb{Q}, \mathbb{R}$ y $\mathbb{C}$ quitando el número cero, son grupos con el producto. Sin embargo, no ocurre así con $\mathbb{Z}^\ast$ ya que no contiene a los inversos.
\item Si $X$ es un conjunto no vacío, el conjunto $Biy(X)$, formado por las aplicaciones $X \longrightarrow X$ que son biyectivas, es un grupo con la operación composición de aplicaciones, cuyo elemento neutro es la aplicación identidad: 
$$\begin{array}{rccl}
1_X \colon &X & \longrightarrow & X\\
&x & \longmapsto &x.
\end{array}
$$

Esto se puede comprobar fácilmente: si $f,g \in Biy(X)$ entonces $(f \circ g)(X) = f(g(X)) = f(X) = X$, lo que prueba la sobreyectividad. Para la inyectividad, si $x,y$ son elementos distintos de $X$, la inyectividad de $g$ nos dice que $g(x) \neq g(y)$, y la de $f$ permite concluir que $f(g(x)) \neq f(g(y))$, y así $f \circ g$ también es inyectiva.
\end{enumerate}
\end{example}

$\hfill \blacksquare$

\begin{definition}[\textbf{\textit{El grupo simétrico $S_n$}}] 
Cuando el conjunto $X$ es finito con $n$ elementos, escribiremos $S_n$ en vez de $Biy(X)$. Este grupo tiene $n!$ elementos, ya que si $X = \lbrace a_1, \ldots, a_n \rbrace$ entonces para definir un elemento en $S_n$ tenemos $n$ posibles valores como imágenes de $a_1$, $n-1$ valores como imagen de $a_2$ (pues al ser biyecciones las imágenes de $a_1$ y $a_2$ deben ser distintas) y, en general, $n-i$ posibles valores como imagen de $a_{i+1}$, con $i = 0, \ldots, n-1$, por lo que el número de elementos de $S_n$ es $$n(n-1) \ldots 3 \cdot 2 \cdot 1 = n!.$$
\end{definition}

\begin{example}\label{ex:gDie} Más ejemplos: 
\begin{enumerate}
\item El conjunto $GL(\mathbb{R})_n$ formado por las matrices de orden $n$ con coeficientes en $\mathbb{R}$ cuyo determinante es no nulo forma un grupo con la operación producto de matrices, con elemento neutro la matriz identidad de orden $n$, $I_n$, y cuyos únicos coeficientes no nulos son los de la diagonal principal, que valen uno.

De hecho, de la fórmula $$det (A \cdot B) = det(A) \cdot det(B)$$ se deduce en particular que el producto es una operación binaria en $GL(\mathbb{R})_n$ y por otro lado, sabemos ya que las matrices con determinante nulo no tienen inversa.

\item (\textbf{\textit{Grupo diédrico}}) Sea $n\geq 3$ un número natural y $X$ el polígono regular de $n$ lados, con vértices $a_1, \ldots, a_n$.

Decimos que una biyección $f \colon X \longrightarrow X$ \textbf{conserva la distancia} si $d(a,b) = d(f(a), f(b))$ para cada $a,b \in X$. Así, llamaremos \textbf{$n$-ésimo grupo diedral} o \textbf{grupo diédrico de orden $n$} al conjunto $$D_n = \lbrace f \in Biy(X) : f~conserva~la~distancia \rbrace.$$

En efecto, es inmediato comprobar que $\mathcal{D}_{n}$ con la operación composición es un grupo: 
\begin{itemize}
\item La asociatividad de $D_n$ se desprende de la asociatividad de $Biy(X)$.
\item La aplicación identidad $1_{X}\in \mathcal{D}_{n}$ (que claramente conserva la distancia) constituye el elemento neutro de $\mathcal{D}_{n}$.
\item Por último, sea $h \in \mathcal{D}_{n}$, y $h^{-1} \in Biy(X)$ la aplicación inversa. Para ver que $h$ posee inversa en $\mathcal{D}_{n}$ basta comprobar que $h^{-1} \in \mathcal{D}_{n}$, es decir, que $h^{-1}$ conserva la distancia. Veámoslo:
\end{itemize}
Sean $a,b \in X$, $p = h^{-1}(a)$, $q = h^{-1}(b)$. Así, $h(p) = a$, $h(q) = b$, y como $h$ conserva la distancia: $$d(a,b) = d(h(p), h(q)) = d(p,q) = d(h^{-1}(a), h^{-1}(b)).$$

Si $f \in \mathcal{D}_{n}$ y $p$ es un punto situado en el segmento que une $a_{i}$ con $a_{i+1}$, se cumple $$d(a_{i}, a_{i+1})= d(a_{i},p) + d(p,a_{i+1}),$$ luego $$d(f(a_{i}), f(a_{i+1}))= d(f(a_{i}),f(p)) + d(f(p),f(a_{i+1})).$$ 

Por lo que $f(p)$ pertenece al segmento que une $f(a_{i})$ con $f(a_{i+1})$. Como $f$ transforma $X$ en $X$ se deduce de lo anterior que envía lados en lados, y por ello, al ser un \textit{vértice} un punto común a dos lados, la imagen por $f$ de un vértice de $X$ es otro vértice de $X$.

Por lo tanto, si $V = \lbrace a_{1}, \ldots, a_{n} \rbrace$, $f|_{V} \in Biy(V)$.

Además, cada $f \in \mathcal{D}_{n}$ queda determinada por las imágenes $f(a_{1}), \ldots, f(a_{n})$ de los vértices, pues dado un $p \in X$, estará entre dos vértices consecutivos $a_{i}$ y $a_{i+1}$, luego $f(p)$ es el único punto del segmento que une $f(a_{i})$ con $f(a_{i+1})$, que dista de éstos lo mismo que $p$ dista de $a_{i}$ y $a_{i+1}$.

Por lo tanto, la aplicación $f \longrightarrow f|_{V}$ entre $\mathcal{D}_{n}$ y $S_{n} = Biy(V)$ es inyectiva. Así que podemos identificar a \textbf{$\mathcal{D}_{n}$ como un subconjunto de $S_{n}$}.

Ahora calculemos los elementos que tiene $\mathcal{D}_{n}$. Si $f \in \mathcal{D}_{n}$ y $f(a_{1}) = a_{i}$, necesariamente será $f(a_{2})= a_{i-1}$ ó $a_{i+1}$, $f(a_{3})=a_{i-2}$ ó $a_{1+2}$, etc. pues $f$ conserva la distancia. En consecuencia, por cada elección de la imagen de $a_{1}$ (y hay sólo $n$ posibles imágenes) tenemos dos modos, a lo sumo, de elegir imagen para el resto de vértices. Por lo tanto, $$|\mathcal{D}_{n}| \leq 2n.$$
Veamos que se da la igualdad, y cuáles son exactamente los elementos de $\mathcal{D}_{n}$.

Si $0$ es el centro del polígono $X$, el giro $f$ de centro $0$ y ángulo $2\pi/ n$ es claro que pertenece a $\mathcal{D}_{n}$ y $f^{n}$ es la identidad. Por lo que $$\lbrace 1_{X} = f^{n}, f, f^{2}, \ldots, f^{n-1} \rbrace \subseteq \mathcal{D}_{n},$$ y estos elementos son distintos ya que $f^{i} = f^{j}$, con $1 \leq i < j \leq n$ implicaría que $$1_{X} = f^{-j}\circ f^{i} = f^{-i} \circ f^{j} = f^{j-i},$$
y así $$a_{1} = 1_{X}(a_{1}) = f^{j-i}(a_{1})= a_{j-i+1},$$
y esto es absurdo.

Por otro lado, la simetría $g$ respecto de la recta que une $0$ con $a_{1}$, es también kelemento de $\mathcal{D}_{n}$ pues conserva la distancia y $g(a_{1}) = a_{1}$, $g(a_{i}) = a_{n-i+2}$, con $2 \leq i \leq n$.

Así que componiendo con las potencias de $f$, tenemos $$\lbrace g, g \circ f, \ldots, g \circ f^{n-1} \rbrace \subseteq \mathcal{D}_{n},$$ y si añadimos lo que ya teníamos $$\lbrace 1_{X}, f, f^{2}, \ldots, f^{n-1}, g, g \circ f, \ldots, g \circ f^{n-1} \rbrace \subseteq \mathcal{D}_{n}.$$
Veamos ahora que todos los elementos son distintos. Si $g \circ f^{i} = g \circ f^{j}$, con $1 \leq i < j \leq n$, tendríamos $$f^{i} = f^{j},$$ que ya hemos visto que es falso. Por otro lado, si $g \circ f^{i} = f^{j}$, con $1 \leq i < j \leq n$ implicaría $$g = f^{j-i},$$ y así $$a_{1} = g(a_{1}) = f^{j-i} (a_{1}) = a_{j-i+1},$$ luego $j-i+1 = 1$ y $j-i = 0$, $g = f^{0} = 1_{X}.$ Entonces $a_{n} = g(a_{2}) = 1_{X}(a_{2}) = a_{2}$, por lo que $n=2$, y esto es imposible.

Al probar que son distintos queda definida la igualdad y así 
\begin{center}
$\mathcal{D}_{n} = \lbrace 1_{X}, f, f^{2}, \ldots, f^{n-1}, g, g \circ f, \ldots, g \circ f^{n-1} \rbrace$ y $|\mathcal{D}_{n}| = 2n$. 
\end{center}

Es por ello que suele escribirse como $\mathcal{D}_{2n}$. Más adelante se dará una descripción alternativa a este grupo.

\end{enumerate}
\end{example}

$\hfill \blacksquare$

\begin{definition}Diremos que un grupo $G$ es \textbf{abeliano} o \textbf{conmutativo} si $ab = ba$ para cada par de elementos $a,b \in G$.
\end{definition}

Es claro que los grupos $\mathbb{Z}, \mathbb{Q}, \mathbb{R}, \mathbb{C}$ con la suma y los grupos $\mathbb{Q}^\ast, \mathbb{R}^\ast, \mathbb{C}^\ast$ con el producto son grupos abelianos.

En particular, todo grupo con dos elementos es claramente abeliano, ya que uno de esos dos elementos tiene que ser necesariamente el elemento neutro, y el otro, denotémoslo por $a$, cumplirá claramente que $ea = ae$ y $aa = aa$.

\begin{proposition}Para $n \geq 3$, $S_n$ no es abeliano.
\end{proposition}
\begin{proof}
En efecto, sea $X = \lbrace 1, 2, \ldots, n \rbrace$ y $S_n = Biy(X)$. Sean $f,g$ elementos de $S_n$ definidos tal que así: $$f(1) = 2, f(2) = 3, f(3) = 1, f(k) = k, k \geq 4$$ $$g(1) = 2, g(2) = 1, g(k) = k, k \geq 3.$$ Como $(g \circ f)(3) = g(1) = 2$, y $(f\circ g)(3) = f(3) = 1$ tenemos que $g \circ f \neq f \circ g$ y $S_n$ no es abeliano.

\end{proof}

\begin{proposition}Para $n \geq 2$, $GL_n(\mathbb{R})$ no es abeliano.
\end{proposition}
\begin{proof}
En efecto, la matriz $A = (a_{ij})$ dada por $$a_{ij} = \left\{
    \begin{array}{ll}
        1 &~si~i \leq j \\
        0 &~si~ i >j
    \end{array}
\right.$$
pertenece a $GL_n(\mathbb{R})$ pues $det(A)= 1 \neq 0$. Como $det A^t = det A$, $A^t \in GL_n(\mathbb{R})$. Ahora, $$AA^t = \left(
\begin{array}{c|c}
    n & \ast \\ \hline
    \ast & \ast \\
\end{array}
\right), \quad A^tA = \left(
\begin{array}{c|c}
    1 & \ast \\ \hline
    \ast & \ast \\
\end{array}
\right),$$
con lo que $AA^t \neq A^tA$.

\end{proof}

\begin{proposition}\label{prop:abDie} Si $n \geq 3$, $D_n$ no es abeliano.
\end{proposition}
\begin{proof}
Si $f$ y $g$ son el giro y la simetría en $D_n$ vistos en el segundo ejemplo de~\ref{ex:gDie} tenemos que $$(g \circ f)(a_1) = g(a_2) = a_n \neq a_2 = f(a_1) = (f\circ g)(a_1).$$

\end{proof}

Veamos otras caracterizaciones de los grupos abelianos:

\begin{proposition}Sea $G$ un grupo. 
\begin{enumerate}
\item Si $x^2 = 1$ para cada $x \in G$, $G$ es abeliano.
\item Si $(ab)^2 = a^2b^2$ para cada $a,b \in G$, $G$ es abeliano.
\end{enumerate}
\end{proposition}
\begin{proof} Veámoslo por partes:
\begin{enumerate}
\item Para cada $x \in G$ se tiene que $x^2 = x \cdot x = 1 = x \cdot x^{-1}$, luego $x = x^{-1}$. Así, sean $a,b \in G$, entonces $a = a^{-1}, b = b^{-1}$ y $$ab = (ab)^{-1} = b^{-1}a^{-1} = ab.$$
\item Sean $a,b \in G$, entonces $$a(ba)b = (ab)^2 = a^2b^2 = aabb = a(ab)b.$$ Así, se tiene que $ab = ba$.
\end{enumerate}

\end{proof}

\begin{definition}[\textbf{\textit{Producto directo}}] Sean $G$ y $G'$ dos grupos, cuyas operaciones notaremos por $$\begin{array}{rccl}
&G \times G & \longrightarrow & G\\
&(a,b) & \longmapsto &ab
\end{array}
$$ 

$$\begin{array}{rccl}
&G' \times G' & \longrightarrow & G\\
&(a',b') & \longmapsto &a'b'
\end{array}
$$

El producto cartesiano $G'' = G \times G'$ es un grupo con la operación $$\begin{array}{rccl}
&G'' \times G'' & \longrightarrow & G\\
&((a,a'),(b,b')) & \longmapsto &(ab, a'b').
\end{array}
$$

La asociatividad en $G''$ es consecuencia inmediata de la asociatividad en $G$ y $G'$ y de que la operación en $G''$ se ha definido elemento a elemento. Evidentemente, el elemento $1_G'' = (1_G, 1_G')$ es el neutro en $G''$. 

Por último, como $$(a,b')(a^{-1}, b'^{-1}) = (aa^{-1}, b'b'^{-1}) = (1_G, 1_G') = 1_G''$$ $$(a^{-1}, b'^{-1}) (a,b') = (a^{-1}a, b'^{-1}b')=(1_G, 1_G') = 1_G'',$$ el elemento $(a^{-1}, b'^{-1})$ es el inverso, en $G''$, de $(a,b')$.

Notar además que si $G$ y $G'$ son abelianos, también lo es $G''$. Recíprocamente, si $G''$ es abeliano, dados $a,b \in G$ se tiene que $$(a, 1_G') (b,1_G') = (b, 1_G')(a, 1_G')$$ y así $ab = ba$, luego $G$ abeliano. Análogo con $G'$.

En general, dados grupos $G_1, \ldots, G_r$, definimos por recurrencia $$G_1 \times \ldots \times G_r = (G_1 \times \ldots \times G_{r-1}) \times G_r.$$ Y diremos que $G_1 \times \ldots \times G_r$ es el \textbf{producto directo} de los grupos $G_1, \ldots, G_r$.
\end{definition}

\subsection{Subgrupos}

\begin{definition}Un subconjunto no vacío $H$ de un grupo $G$ es un \textbf{subgrupo} de $G$ si con la misma operación de $G$ es un grupo.
\end{definition}

\begin{proposition}Sea $H$ un subgrupo de un grupo $G$, entonces:
\begin{enumerate}
\item $1_G$ pertenece a $H$ y es su elemento neutro.
\item Si $x \in H$, también $x^{-1} \in H$.
\end{enumerate} 
\end{proposition}
\begin{proof}
Veamos:
\begin{enumerate}
\item Por definición $H$ tiene un elemento neutro al que llamamos $e$. Desde luego, $ee = e$. Sea $e^{-1} \in G$ el inverso de $e$ en $G$. Así, operando en $G$ tenemos que $e^{-1}(ee) = e^{-1}e = 1_G$, luego $(e^{-1}e)e = 1_G$ y así $1_Ge =1_G$, o sea, $e = 1_G$.
\item Si $x \in H$ entonces existe $y \in H$ tal que $xy = 1_G = yx$, ya que sabemos que $1_G$ es el elemento neutro de $H$. Así, $xx^{-1} = xy$ y aplicando la propiedad cancelativa $x^{-1} = y \in H$.
\end{enumerate}

\end{proof}

La siguiente proposición es la que se suele usar como caracterización usual de los subgrupos.

\begin{proposition}\label{prop:carSub} Sea $H$ un subconjunto no vacío de un grupo $G$. Las siguientes condiciones son equivalentes:
\begin{enumerate}
\item $H$ es un subgrupo de $G$.
\item Para cada par de elementos $x,y \in H$, $xy^{-1} \in H$.
\end{enumerate}
\end{proposition}
\begin{proof}
Veamos: 

$1. \Rightarrow 2.$ Dados dos elementos $x,y \in H$, sabemos que entonces $y ^{-1} \in H$. El producto es una operación binaria en $H$, porque $H$ es subgrupo. Así, como $x, y^{-1} \in H$, se sigue que $xy^{-1} \in H$.

$2. \Rightarrow 1.$ Sea $x \in H$ (existe ya que $H$ es no vacío). Ahora, si tomamos $y = x$ tenemos que $xx^{-1} \in H$ y así $1_{G} \in H$, luego $H$ tiene elemento neutro. Ahora, dado un $y \in H$, si tomamos $x = 1_{G} \in H$, tenemos que $y^{-1} = 1_{G}y^{-1} = xy^{-1} \in H$ luego cada elemento de $H$ tiene inverso en $H$. Finalmente, dados $x,y \in H$ ya sabemos que $z = y^{-1} \in H$, luego $xy = x(y^{-1})^{-1} = xz^{-1} \in H$ y así la operación de $G$ es una operación binaria de $H$. La asociatividad es evidente, pues lo es para cada terna de elementos de $G$.

\end{proof}

Notar que en la proposición se ha usado la notación multiplicativa, en el caso de que estuviésemos usando una aditiva sería $x-y \in H$ en lugar de $xy^{-1} \in H$.

\begin{observation}Evidentemente, $\lbrace 1_G \rbrace$ y $G$ son subgrupos de cualquier grupo $G$. Llamaremos \textbf{subgrupos propios} de $G$ a aquellos subgrupos distintos de $\lbrace 1_G \rbrace$ y $G$.
\end{observation}

\begin{example}
Los subgrupos de $\mathbb{Z}$ son de la forma $$m\mathbb{Z} = \lbrace mx:x \in \mathbb{Z} \rbrace$$ para cada entero no negativo $m$. 

Desde luego $m\mathbb{Z}$ es un subgrupo de $\mathbb{Z}$, pues es no vacío ya que $m = m1 \in m\mathbb{Z}$, y si $a = mx$, $b = my$ pertenecen a $m\mathbb{Z}$, $a-b = mx-my=m(x-y) \in m\mathbb{Z}$. Por~\ref{prop:carSub} $m\mathbb{Z}$ es un subgrupo de $\mathbb{Z}$.

Recíprocamente, sea $H$ un subgrupo de $\mathbb{Z}$. Si $H$ consta sólo del número cero, $H = 0 \mathbb{Z}$ tiene la forma requerida. Si $H$ tiene algún elemento no nulo, tiene necesariamente alguno positivo ya que $x^{-1} \in H$ para un $x \in H$. Si $m$ es el menor entero positivo en $H$, cualquier otro $n \in H$ positivo será $$n = qm+r, \quad 0 \leq r < m.$$ Como $qm = m\underbrace{+ \ldots + }_q m \in H$, $r = n-qm \in H$ y es menor que $m$, luego por la elección de $m$, no es positivo. Así, $r = 0$ y por lo tanto $n = qm = mq \in m\mathbb{Z}$. Igualmente, si $n \in H$ es negativo, $-n \in H$ es positivo, luego $-n = mx \in m \mathbb{Z}$ para algún entero $x$. Así, $n = m(-x) \in m \mathbb{Z}$. Como también $0 = m0 \in m \mathbb{Z}$, tenemos que $H \subseteq m\mathbb{Z}$. Pero como $m \in H$ también $-m \in H$, y así para cada $x \in \mathbb{Z}$ tenemos: 
$$
mx = \left\{
    \begin{array}{ll}
        m \underbrace{+ \ldots +}_x m \in H &~si~ x>0 \\
        0 \in H &~si~ x=0 \\
        (-m) \underbrace{+ \ldots +}_x (-m) \in H&~si~ x<0
    \end{array}
\right.
$$
con lo que $H = m\mathbb{Z}$.
\end{example}

$\hfill \blacksquare$

Uno de los modos habituales de construir grupos es: 

\begin{definition}Si $S$ es un subconjunto no vacío de un grupo $G$, el conjunto $$\langle S \rangle = \lbrace s_1^{h_1} \ldots s_n^{h_n}: n \in \mathbb{N}, s_i \in S, h_i \in \mathbb{Z}, 1 \leq i \leq n \rbrace$$ es un subgrupo de $G$ que contiene a $S$, llamado \textbf{subgrupo generado por $S$}.
\end{definition}

\begin{observation}Dado $S$ un subconjunto no vacío de un grupo $G$, entonces, $$\langle S \rangle = \lbrace x_1 \ldots x_m: m \in \mathbb{N}, x_i \in S,~\acute{o}~ x_i^{-1} \in S, 1 \leq i \leq m \rbrace.$$ Además, si $\mathcal{F}_S$ es la familia de todos los subgrupos de $G$ que contienen a $S$, $$\langle S \rangle = \bigcap_{H \in \mathcal{F}_S} H.$$ En particular, $\langle S \rangle \subseteq H$ para cada $H \in \mathcal{F}_S$.
\end{observation}
\begin{proof}
Cada $s \in S$ se escribe $s = s^1 \in \langle S \rangle$. Esto prueba que $S \subseteq \langle S \rangle$. En particular $\langle S \rangle$ es no vacío, por no serlo $S$.

Dados $x = s_1^{h_1} \ldots s_n^{h_n}$, $y = t_1^{l_1}\ldots t_m^{l_m}$, con $x,y \in \langle S \rangle$. Como $y^{-1}= t_m ^{-l_m} \ldots t_1^{-l_1}$ tenemos $$xy^{-1} = s_1^{h_1} \ldots s_n^{h_n}t_m ^{-l_m} \ldots t_1^{-l_1} \in \langle S \rangle.$$ Queda probado así que $\langle S \rangle$ es un subgrupo de $G$.

Ahora, dado $x = x_1 \ldots x_m, m \in \mathbb{N}$, $x_i \in S$ ó $x_i^{-1} \in S$, consideramos, para cada $1 \leq i \leq n$ 
$$
s_i = \left\{
    \begin{array}{ll}
        x_i&~si~ x_i \in S \\
       x_i^{-1}&~si~ x_i^{-1} \in S
    \end{array}
\right.
$$
$$
h_i = \left\{
    \begin{array}{ll}
       1&~si~ x_i \in S \\
       -1&~si~ x_i^{-1} \in S
    \end{array}
\right.
$$

Evidentemente, para cada $1 \leq i \leq n$, $s_i \in S$, $s_i^{h_i} = x_i$. Así, $x = s_1^{h_1} \ldots s_n^{h_n} \in \langle S \rangle$, luego $$\lbrace x_1 \ldots x_m: m \in \mathbb{N}, x_i \in S,~\acute{o}~ x_i^{-1} \in S, 1 \leq i \leq m \rbrace \subseteq \langle S \rangle.$$

Recíprocamente, sea $x =  s_1^{h_1} \ldots s_n^{h_n} \in \langle S \rangle$, $n \in \mathbb{N}$, $s_i \in S$, $h_i \in \mathbb{Z}$, $1 \leq i \leq n$. Como $s_i^0 = 1$, podemos suponer que cada $h_i \neq 0$. Consideremos, para cada $1 \leq i \leq n, $ 

$$
l_i = \left\{
    \begin{array}{ll}
        h_i&~si~ h_i >0 \\
       	-h_i&~si~ h_i <0
    \end{array}
\right.
$$
 y fijado $i$ pongamos para cada $1 \leq k \leq l_i,$ 
 
$$
x_{ki} = \left\{
    \begin{array}{ll}
        s_i&~si~ h_i >0 \\
       	s_i^{-1}&~si~ h_i <0
    \end{array}
\right.
$$ 

Desde luego, para cada $1 \leq i \leq n$, $1 \leq k \leq l_i$, bien $x_{ki}  \in S$ (si $h_i >0$), bien $x^{-1}_{ki} = s_i \in S$ (si $h_i <0$). Además, $s_i ^{h_i} = x_{1i} \ldots x_{l_ii}$, $1 \leq i \leq n$, luego $$x = x_{11} \ldots x_{l_11} \ldots x_{1n} \ldots x_{l_nn}$$ pertenece a $\lbrace x_1 \ldots x_m: m \in \mathbb{N}, x_i \in S,~\acute{o}~ x_i^{-1} \in S, 1 \leq i \leq m \rbrace \subseteq \langle S \rangle$, y así tenemos la igualdad.

Finalmente, ya sabemos que $\langle S \rangle \in \mathcal{F}_S$, de donde $$ \bigcap_{H \in \mathcal{F}_S} H \subseteq \langle S \rangle.$$ Para probar la igualdad bastará pues ver que $\langle S \rangle \subseteq H$ para cada $H \in \mathcal{F}_S$. Dado $x = s_1^{h_1} \ldots s_n^{h_n} \in \langle S \rangle$, cada $s_i \in S \subseteq H$ y al ser $H$ subgrupo, también $s_i^{hi} \in H$, de donde $x \in H$.

\end{proof}

\begin{definition} Un caso particular pero muy importante es aquel en el que $S = \lbrace a \rbrace$ para algún $a \in G$. En tal caso escribiremos $\langle a \rangle$ en vez de $\langle \lbrace a \rbrace \rangle$. Es claro que $$ \langle a \rangle = \lbrace a^k : k \in \mathbb{Z} \rbrace$$ y se le llama \textbf{subgrupo generado por $a$}.
\end{definition}

\begin{definition} Un subconjunto no vacío $S$ de un grupo $G$ se llama \textbf{sistema generador} de $G$ si $G  =\langle S \rangle$.

Como el menor subgrupo de $G$ que contiene a $G$ es el propio $G$, deducimos que $\langle G \rangle = G$, luego $G$ es un sistema generador de $G$.
\end{definition}

\begin{example}En el caso del grupo diédrico tenemos que $$D_n = \lbrace 1, f, \ldots, f^{n-1}, g, g \circ f, \ldots, g \circ f^{n-1}\rbrace$$ y así $D_n = \langle S \rangle$, con $S = \lbrace f,g \rbrace$.
\end{example}

$\hfill \blacksquare$

\begin{definition}Un grupo $G$ que posee un sistema finito de generadores diremos que es \textbf{finitamente generado}. Así, todo grupo finito $G$ es finitamente generado porque, tal y como se ha visto, $G$ es un sistema generador de $G$.
\end{definition}

Sin embargo, el recíproco en general no es cierto. Por ejemplo, el grupo $\mathbb{Z}$ de los números enteros está generado por $\lbrace 1 \rbrace$, ya que, dado un $n \in \mathbb{Z}$, 
$$
n = \left\{
    \begin{array}{ll}
       1 \underbrace{+ \ldots +}_n1&~si~ n>0 \\
       	(-1) \underbrace{+ \ldots + }_n (-1)&~si~ n<0
    \end{array}
\right.
$$ 
Sin embargo, es claro que $\mathbb{Z}$ no es finito.

\begin{observation}Aunque obvio, lo siguiente es con frecuencia útil. Y es que dado un grupo $G$, y dos subconjuntos $S$ y $S'$ de $G$, para que los subgrupos $H = \langle S \rangle$ y $K = \langle S' \rangle$ coincidan es suficiente que $S \subseteq K$ y $S' \subseteq H$, pues en tal caso si $x \in H$ será de la forma $x = s_1 \ldots s_m$, con $s_i \in S \subseteq K$, luego como $K$ es subgrupo, $x \in K$ y hemos probado $H \subseteq K$.

Recíprocamente, cada $x \in K$ se escribe como $x = s'_1 \ldots s'_n$, con $s'_i \in S' \subseteq H$, luego $x \in H$, es decir, $K \subseteq H$.
\end{observation}

\begin{definition}Si $H$ es un subgrupo de $G$, llamaremos \textbf{centralizador} de $H$ en $G$ a $$C_G(H) = \lbrace x \in G : ax = xa \hspace{0.2cm} \forall a \in H \rbrace.$$

Al centralizador de $G$ en $G$ lo denotaremos por $Z(G)$ y se le denominará \textbf{centro} de $G$. Evidentemente $$Z(G) = \lbrace x \in G: ax = xa \hspace{0.2cm} a \in G \rbrace,$$ y por lo tanto $G$ es abelianos si y sólo si $G = Z(G)$.
\end{definition}

\begin{observation}El centro es un subgrupo de $G$. De hecho, $C_G(H)$ es subgrupo de $G$.
\end{observation}
\begin{proof}
Como $1_{G} \in C_{G}(H),$ éste no es vacío. Sean $x,y \in C_{G}(H),\hspace{0.1cm} a \in H$. Como $x\in C_{G}(H),\hspace{0.1cm} ax = xa$. Como $y \in C_{G}(H), \hspace{0.1cm} a^{-1} \in H,\hspace{0.1cm} a^{-1}y = ya^{-1}$. Por lo tanto, \begin{center}$a(xy^{-1}) = (ax)y^{-1} = (xa)y^{-1} = x(ay^{-1}) = x(ya^{-1})^{-1} = x(a^{-1}y)^{-1} = x(y^{-1}a) = (xy^{-1})a$\end{center} luego $xy^{-1} \in C_{G}(H)$. Así, $C_{G}(H)$ es un subgrupo de $G$.

\end{proof}

\begin{observation}En el caso particular de que $H = \langle a \rangle$ para algún $a \in G$, entonces $x \in C_G(H)$ si y sólo si $xa = ax.$
\end{observation}
\begin{proof}
En efecto, el sólo si es obvio, pues $a \in H$. Para probar el si tenemos que ver que $ax = xa$ implica $a^kx = xa^k$ para cada $k \in \mathbb{Z}$. 

Lo haremos por inducción sobre $k$. Si $k=1$ no hay nada que probar. Si $k>1$, $$a^kx = a(a^{k-1}x) = a(xa^{k-1}) = (ax)a^{k-1} = (xa)a^{k-1} = xa^k$$ donde hemos usado la hipótesis de inducción en la segunda y cuarta igualdades. Antes de abordar el caso $k<0$, observemos que de $ax = xa$ se deduce $a^{-1}(ax)a^{-1} = a^{-1}(xa)a^{-1}$, luego $xa^{-1} = a^{-1}x$. Ahora, si $k = -l < 0,$ con $l \in \mathbb{N}$, probaremos por inducción sobre $l$ que $(a^{-1})^lx = x(a^{-1})^l$, de donde $a^kx = xa^k$. Para $l = 1$ no hay nada que probar, y si $l>1$ $(a^{-1})^lx = a^{-1}(a^{-1})^{l-1}x = a^{-1}x (a^{-1})^{l-1} = xa^{-1}(a^{-1})^{l-1} = x(a^{-1})^l.$

\end{proof}

Con esto, tenemos $$C_G(\langle a \rangle) = \lbrace x \in G : ax = xa \rbrace.$$ Por eso se suele esccribir $C_G(a)$ en lugar de $C_G(\langle a \rangle)$. Evidentemente es claro que $Z(G) = \cap_{a \in G} C_G(a)$. Además

\begin{observation}$a \in Z(G)$ si y sólo si $C_G(a) = G$.
\end{observation}
\begin{proof}
Si $a \in Z(G)$ cada $x\in G$ cumple $ax = xa$, luego $G \subseteq C_G(a) \subseteq G$. Recíprocamente, si $C_G(a) = G$ cada $x \in G$ pertenece a $C_G(a)$, luego $ax = xa$ para cada $x \in G$ y así $a \in Z(G)$.

\end{proof}

\begin{proposition}Si $S$ es un subconjunto no vacío de un grupo $G$ y $a \in G$, llamaremos \textbf{conjugado de $S$ por $a$} al conjunto $$S^a = \lbrace a^{-1}xa: x \in S \rbrace.$$

Además, es claro que $y \in S^a$ si y sólo si $aya^{-1} \in S$.
\end{proposition}

\begin{properties}Algunas propiedades del conjugado son: 
\begin{enumerate}
\item Se tiene que $$\begin{array}{rccl}
&S & \longrightarrow & S^a\\
&x & \longmapsto &a^{-1}xa
\end{array}
$$ es biyectiva.
\item $(S^a)^b = S^{ab}$ para cualesquiera $a,b \in G$.
\item $S = S^1$.
\item Si $S$ es subgrupo de $G$, también lo es $S^a$.
\item Si $S \subseteq T$, entonces $S^a \subseteq T^a$.
\end{enumerate}
\end{properties}
\begin{proof}
Veamos: 
\begin{enumerate}
\item Basta ver la inyectividad. Pero si $a^{-1}xa = a^{-1}ya$, se sigue que $xa = ya$ y de aquí $x=y$.
\item Como $z \in (S^a)^b$ equivale a $z = b^{-1}yb$, $y \in S^a$ y esto es lo mismo que $z = b^{-1}yb$, $y = a^{-1}xa$, con $x \in S$ entonces $$z = b^{-1}(a^{-1}xa)b = (b^{-1}a^{-1})x(ab) = (ab)^{-1}x(ab) \in S^{ab}.$$
\item Simplemente, si $x \in S$, entonces $1^{-1}x1 = 1x1 = x$.
\item Cuando $S$ es subgrupo, $1 \in S$ y así $a^{-1}1a \in S^a$, esto es, $1 \in S^a$. Así, $S^a$ es no vacío. Además, dados $u,v \in S^a$ serán $u = a^{-1}xa$, $v = a^{-1}ya$ para algunos $x,y \in S$, y por lo tanto $uv^{-1} = a^{-1}xa(a^{-1}ya)^{-1} = a^{-1}xa a^{-1}y^{-1}a = a^{-1}xy^{-1}a \in S^a$ por ser $S$ subgrupo de $G$ (y así $xy^{-1} \in S$).
\item Si $x \in S^a$ tenemos que $axa^{-1} \in S \subseteq T$, luego $x \in T^a$.
\end{enumerate}

\end{proof}

\begin{definition}Si $S$ es un subconjunto no vacío de un grupo $G$, llamaremos \textbf{normalizador} de $S$ en $G$ a $$N_G(S) = \lbrace a \in G : S^a = S \rbrace,$$ que además es un subgrupo de $G$.
\end{definition}
\begin{proof}
Veamos que es subgrupo. Ya sabemos que $S = S^1$, luego $1 \in N_G(S)$ y así $N_G(S)$ es no vacío. Por otro lado, si $a,b \in N_G(S)$ tenemos $S^{ab^{-1}} = (S^a)^{b^{-1}} = S^{b^{-1}}$ ya que $a \in N_G(S)$. Como $S = S^1 = S^{bb^{-1}} = (S^b)^{b^{-1}} = S^{b^{-1}}$, ya que $b \in N_G(S)$, tenemos entonces que $$S^{ab^{-1}} = S,$$ y así $ab^{-1} \in N_G(S)$.

\end{proof}

\begin{observation}\label{ob:intGru} Si $\lbrace H_i : i \in I \rbrace$ es una familia no vacía de subgrupos de un grupo $G$, entonces $$H = \bigcap_{i \in I} H_i$$ es un subgrupo de $G$. Además, para cada $a \in G$ se tiene que $$H ^a = \bigcap_{i \in I} H_i^a.$$
\end{observation}
\begin{proof}
Esto es así puesto que $1 \in H$ y si $x,y \in H$ se sigue que $x,y \in H_i$ para cada $i \in I$ y así $xy^{-1} \in H_i$, por ser $H_i$ subgrupo, para cada $i \in I$. Por lo tanto, $xy^{-1} \in H$.

Para lo segundo, si $x \in H^a$ entonces $axa^{-1} \in H$, luego $axa^{-1} \in H_i$ para cada $i \in I$, o lo que es lo mismo, $x \in H_i^a$ para cada $i \in I$. Así, $ H^a \subseteq \cap_{i \in I}H_i^a.$ Recíprocamente, si $x \in \cap_{i \in I} H_i^a$ se tiene que $x \in H_i^a$ para todo $i \in I$, o sea, $axa^{-1} \in H_i$ para todo $i \in I$ y así $$axa^{-1} \in \bigcap_{i \in I} H_i = H,$$ de donde $x \in H^a$. Esto prueba $\cap_{i \in I} H_i^a \subseteq H^a$ y así la igualdad.

\end{proof}

\begin{definition}[\textbf{\textit{Grupo producto}}]Dados dos subgrupos $H$ y $K$ de un grupo $G$, definimos $$HK = \lbrace hk: h \in H, k \in K \rbrace.$$ 
\end{definition}

Sin embargo, este producto no se suele comportar muy bien. En general, el producto de subgrupos no será subgrupo, para que lo sea tendrá que ocurrir lo siguiente:

\begin{proposition}$HK$ es subgrupo de $G$ si y sólo si $HK = KH$. Es claro que $H \subseteq HK$, $K \subseteq HK$.
\end{proposition}
\begin{proof}
Supongamos que $HK$ es subgrupo de $G$. Si $x = hk \in HK$ entonces $k^{-1}h^{-1} = x^{-1} \in HK$, luego $k^{-1}h^{-1} = uv$ con $u \in H$, $v \in K$ y así $x = hk = (k^{-1}h^{-1})^{-1} = (uv)^{-1} = v^{-1}u^{-1} \in KH$ y esto prueba $HK \subseteq KH$. Sea ahora $y = kh \in KH$. Entonces $z = h^{-1}k^{-1} \in HK$, y como $HK$ es subgrupo $y = kh = (h^{-1}k^{-1})^{-1} = z^{-1} \in HK$, y así $KH \subseteq HK$.
 
Recíprocamente, supongamos que $HK = KH$. Evidentemente $HK$ es no vacío, pues $1 = 1 \cdot 1 \in HK$. Además, dados $x = h_{1}k_{1}$, $y = h_{2}k_{2}$, con $x,y \in HK$,$xy^{-1} = h_{1}k_{1}k_{2}^{-1}h_{2}^{-1} = h_{1}k_{3}h_{2}^{-1}$, con $k_{3} = k_{1}k_{2}^{-1} \in K$. Como $k_{3}h_{2}^{-1} \in KH = HK$, $k_{3}h_{2}^{-1} = h_{3}k$, con $h_{3} \in H$, $k \in K$. Así, $xy^{-1} = h_{1}h_{3}k = hk \in HK$, con $h = h_{1}h_{3} \in H$.

\end{proof}

\begin{example}\label{ex:idBez} Sean $m$ y $n$ enteros no negativos, $H = m\mathbb{Z}$, $K = n\mathbb{Z}$ dos subgrupos de $\mathbb{Z}$. Como $\mathbb{Z}$ es abeliano es obvio que $H+K = K+H$, luego por el resultado anterior $H+K$ es subgrupo de $\mathbb{Z}$ (notar que aquí la operación es la suma).


$H+K$ no es el subgrupo $\lbrace 0 \rbrace$ pues, $m = m + 0 \in H + K$. Y, como ya sabemos, existirá un $d \in \mathbb{Z}$ tal que $m\mathbb{Z}+n\mathbb{Z} = d\mathbb{Z}$, veamos que $d = mcd(m,n)$:


Como $m = m + 0 \in m\mathbb{Z}+n\mathbb{Z} = d\mathbb{Z}$, $d$ divide a $m$, y como $n = 0 + n \in m\mathbb{Z}+n\mathbb{Z} = d\mathbb{Z}$, $d$ divide a $n$. Además $d \in d\mathbb{Z} = m\mathbb{Z} + n\mathbb{Z}$ luego existen $a,b \in \mathbb{Z}$, tal que $d = ma + nb$. Entonces, dado un $c$ que divida a $m$ y $n$:
$$m = cu,\hspace{0.1cm} n = cv, u, u \in \mathbb{Z}$$ tenemos $d = (cu)a + (cv)b = c(ua + vb)$ y $c$ divide a $d$. Esto prueba que $d = mcd(m,n)$. 

En particular, dos números enteros $m,n$ son primos entre sí si y sólo si $$1 = am + bn \hspace{0.2cm} a,b \in \mathbb{Z}.$$ En efecto, si $mcd(m,n) = 1$, es $m\mathbb{Z} + n\mathbb{Z} = 1\mathbb{Z}$ por lo visto ahora. Así, $1 \in m\mathbb{Z}+n\mathbb{Z}$ y existirán $a,b \in \mathbb{Z}$ tales que $1 = am + bn$. Recíprocamente, si $1 = am + bn$ y $d$ es un divisor de $m$ y $n$, tendremos $m = du$, $n = dv$, luego $1 = d(au + bv)$ y así $d = +1$ ó $-1$. Y como podemos asumir que $mcd(m,n)$ es positivo entonces $mcd(m,n) = 1$.
\end{example}

$\hfill \blacksquare$

\begin{observation} Dados dos subgrupos $H$ y $K$ de un grupo $G$ tales que $H \subseteq K$ se tiene $HK = K = KH$.

En efecto, cada $x \in HK$ se escribe como $x=hk$, con $h \in H \subseteq K$ y $k \in K$ y así $HK = KH \subseteq K$. Recíprocamente, cada $k \in K$ es de la forma $k  = 1 \cdot k \in HK$, y así $K \subseteq HK$. Análogo con $KH$.
\end{observation}

Otra noción importante de un grupo es el número de elementos que tiene, su cardinal si lo vemos como conjunto.  Aunque no es exactamente lo mismo, veremos que en algunos grupos podremos tener todos los elementos que queramos pero el orden no será infinito, como es el caso de los \textit{grupos cíclicos}. Además, también vamos a ver cómo extender este concepto a un sólo elemento cualquiera de un grupo $G$ cualquiera, y tendrá una íntima relación con el subgrupo que genera.

\subsection{Orden de un grupo}

\begin{definition}Sea $G$ un grupo. Al número de elementos de un subgrupo finito $H$ de $G$ se le llama \textbf{orden} de $H$ y lo notaremos por $o(H)$. En particular, cuando $G$ es finito, el número de elementos de $G$ se llama orden de $G$. En caso contrario, diremos que $G$ es un grupo infinito.

Un elemento $a \in G$ se llama de \textbf{torsión} si el subgrupo $\langle a \rangle$ es finito. En tal caso llamaremos \textbf{orden de $a$} y lo denotaremos por $o(a)$ al orden del subgrupo $\langle a \rangle$.
\end{definition}

Es decir, hemos definido también el orden de un elemento como $o(a) = o(\langle a \rangle)$.

\begin{example}Tanto $\mathbb{Z}$ como todos sus subgrupos son grupos infinitos. Sin embargo, para cada $n \geq 2$, $o(S_n) = n!$ y, para cada $n \geq 3$, $o(D_n) = 2n$.
\end{example}

$\hfill \blacksquare$

Veamos algunas propiedades interesantes del orden y algunos resultados importantes: 

\begin{proposition}\label{prop:ordenes} Sea $G$ un grupo y $a \in G$ un elemento de torsión (su subgrupo generado es finito). Entonces:
\renewcommand{\labelenumi}{\arabic{enumi}.}
\begin{enumerate}
\item Existe $k \geq 1$ tal que $a^{k}=1$.
\item El orden de $a$ es el menor natural $n \geq 1$ tal que $a^{n} = 1$
\item Si $n = o(a)$, entonces $\langle a \rangle = \lbrace 1, a, \cdots, a^{n-1} \rbrace.$
\item Si $n = o(a)$ y $k \in \mathbb{N}$, $a^{k} = 1$ si y sólo si $k$ es múltiplo de $n$.($n$ divide a $m$).
\item $o(a)=1$ si y sólo si $a=1$.
\item $a^{-1}$ es un elemento de torsión y $o(a^{-1}) = o(a)$.
\item Si $x = a^k \in \langle a \rangle$ y $o(a) = n$, $x$ es de torsión y $$o(x) = \dfrac{n}{mcd(n,k)}.$$ 
\item Si $b \in G$ es otro elementos de torsión y $ab = ba$, entonces $ab$ es de torsión y $o(ab)$ es un divisor del $mcm(o(a), o(b))$, con $mcm$ el mínimo común múltiplo.
\item En el punto anterior, si $o(a)$ y $o(b)$ son primos entre sí, $o(ab) = o(a)o(b)$.
\item Si $b \in G$ y $ab$ es de torsión, también lo es $ba$, y $o(ab) = o(ba)$.
\end{enumerate}
\end{proposition}
\begin{proof}Lo veremos por partes: 
\begin{enumerate}
\item Como $\langle a \rangle$ es finito, la aplicación 
$$
\begin{array}{rccl}
&\mathbb{N} \backslash \lbrace 0 \rbrace & \longrightarrow &\langle a \rangle\\
&m& \longmapsto &a^{m}
\end{array}
$$ no es inyectiva. Así, existen $r<s \in \mathbb{N}$ tales que $a^{r} = a^{s}.$ Si $k = s - r$, $1 = a^{0} = a^{r}a^{-r} = a^{s}a^{-r} = a^{s-r} = a^{k}$ .
\item Sea $n$ el menor natural que cumple $a^{n} = 1$, cuya existencia se deduce de lo que acabamos de demostrar en el punto anterior. Si probamos que $$\langle a \rangle = \lbrace 1, a, \ldots, a^{n-1} \rbrace$$ y que todos los elementos del miembro de la derecha son distintos, entonces tendremos que $o(a) = n$.
Evidentemente el elemento de la izquierda de la igualdad contiene al de la derecha. Recíprocamente, si $x = a^{k}$, $k \in \mathbb{Z}$, dividimos por $n$ y por el algoritmo de la división sabemos que: $$k = qn + r, \hspace{0.1cm} 0\leq r \leq n-1,$$ luego $x = a^{qn+r}= (a^{n})^{q}a^{r} = 1^{q}a^{r} = a^{r}, \hspace{0.1cm}  0\leq r \leq n-1.$ Por último, si existieran $0\leq r < s \leq n-1$ tales que $a^{r} = a^{s},$ sería $a^{s-r} = a^{s}a^{-r} = a^{r}a^{-r} = a^{0} = 1, \hspace{0.1cm} s-r \leq n-1 < n,$ pero esto es absurdo porque hemos definido a $n$ como el menor natural poitivo tal que $a^{n} = 1$.
\item Queda demostrado con lo visto en el punto anterior.
\item Si $k = nm$ es múltiplo de $n$, $a^{k} = a^{nm} = (a^{n})^{m} = 1.$ Recíprocamente, si $k$ no es múltiplo de $n$, $k = nm + r, \hspace{0.1cm} 1\leq r\leq n-1$, luego $a^{k} = a^{nm + r} = (a^{n})^{m}a^{r} = 1^{m}a^{r} = a^{r} \neq 1$ por 2.
\item Si $o(a) =1$, $a = a^1 = 1$ por $2.$ Recíprocamente, como $1^1 = 1$, $o(1) = 1$.
\item $\langle a \rangle = \langle a^{-1} \rangle$ puesto que $a^k = (a^{-1})^{-k}$ para cada entero $k$. Así, $o(a) =$ número de elementos de $\langle a \rangle =$ número de elementos de $\langle a^{-1} \rangle = o(a^{-1})$ y $a^{-1}$ es de torsión.
\item Como $x = a^k$ cada elemento $y=a^l$ de $\langle x \rangle$ cumple que $y=(a^k)^l = a^{kl} \in \langle a \rangle$, luego $\langle x \rangle \subseteq \langle a \rangle$ es finito y así $x$ es de torsión.

Sea ahora $d = mcd(n,k)$. Como $d$ divide a $k$, tenemos $k=ed$, para algún $e \in \mathbb{Z}$. Así, $x^{n/d} = a^{kn/d} = a^{ne} = (a^n)^e = 1^e=1$ luego $n/d$ es múltiplo de $o(x)$. Por otro lado, $a^{ko(x)} = (a^k)^{o(x)} = x^{o(x)} = 1$ y así $ko(x)$ es múltiplo de $n$. Entonces, podemos expresar $ko(x) = nm$ para un cierto $m \in \mathbb{Z}$, esto es, $$k = m \dfrac{n}{o(x)},~es~decir~, \dfrac{n}{o(x)}~divide~a~k.$$ Como evidentemente $n/o(x)$ divide a $n$, $n/o(x)$ divide a $d = mcd(n,k)$, es decir, $l n/o(x) = d$ para algún $l \in \mathbb{Z}$. En consecuencia, $ln/d = o(x)$ y así $o(x)$ también es múltiplo de $n/d$. Por lo tanto, $o(x) = n/d$.
\item Sean $n = o(a), m=o(b)$, $M = pn = qm$, $p,q \in \mathbb{N}$, $M = mcm(m,n)$. Como $ab = ba$, tenemps $$(ab)^M = a^Mb^M = (a^n)^p(b^m)^q = 1^p1^q = 1$$ y $o(ab)$ divide a $M$ por lo que vimos en el punto $4.$
\item Como $o(a) = n$ y $o(b) = m$ son primos entre sí, el mínimo común múltiplo de $m$ y $n$ es $M = nm$. Por el punto $8.$ $o(ab)$ divide a $nm$. Llamemos $s$ al orden de $ab$. Así, $(ab)^s = 1$ y como $ab = ba$, $a^sb^s=(ab)^s = 1$, luego $a^s = b^{-s}$. En particular, $o(a^s) = o(b^{-s}) = o((b^s)^{-1}) = o(b^s)$ donde para la último igualdad se ha utilizado el punto $6.$ Ahora, por $7.$, $$\dfrac{n}{mcd(n,s)} = o(a^s) = o(b^s) = \dfrac{m}{mcd(m,s)}.$$ Así, $$d = \dfrac{n}{mcd(n,s)} = \dfrac{m}{mcd(m,s)}~divide~a~m~y~a~n,$$ luego $d=1$, es decir, $n = mcd(n,s)$, $m = mcd(m,s)$. Entonces, $s$ es múltiplo de $n$ y de $m$, luego lo es de $M = nm$. Con esto hemos probado que $o(ab) = s = nm = o(a)o(b)$.
\item Sea $n = o(ab)$. Tendremos $a(ba)^{n-1}b = a(baba...ba)b = (ab)^n = 1,$ luego $(ba)^{n-1}b = a^{-1}$, $(ba)^{n-1} = a^{-1}b^{-1} = (ba)^{-1}$, de donde $(ba)^n = (ba)^{n-1}ba = (ba)^{-1}ba = 1,$ y así el orden de $ba$ divide al de $ba$. Cambiando los papeles de $a$ y $b$ obtenemos que el orden de $ab$ divide al de $ba$, luego $o(ab)=o(ba)$.
\end{enumerate}
\end{proof}

\begin{example}\label{ej:gDie2} Sea $n\geq 3$ y $D_n$ el correspondiente grupo diédrico. Con lo que sabemos, sean $f,g \in D_n$, el giro de ángulo $2\pi/n$ y una simetría respecto de una recta. Si $V = \lbrace a_1, \ldots, a_n \rbrace$ son los vértices del polígono regular de $n$ lados, vimos que $g(a_1) = a_1$ y $g(a_i) = a_{n-i+2}$, con $2 \leq i \leq n$ (es decir, $g$ es la simetría). Como $g(a_2) = a_n \neq a_2$, entonces $g\neq 1$ y así $o(g)>1$. Además, $g^2(a_1) = g(a_1) = a_1$ y $g^2(a_i) = g(a_{n-i+2}) = a_{n-(n-i+2)+2} = a_{i}$, con $2 \leq i \leq n$, luego $g^2 = 1$ y así $o(g) = 2$.

En cuanto a $f$ (el giro), ya sabemos que $f^n = 1$ y que $f^i \neq 1$ si $1 \leq i <n$. Por ello, $o(f) = n$.

\end{example}
$\hfill \blacksquare$

\subsection{Índice de un subgrupo y Teorema de Lagrange}

\begin{definition}Sea $G$ un grupo y $H$ un subgrupo de $G$. Llamaremos $R_{H}$ y $R^{H}$ a las siguientes \textit{relaciones} en $G$:
\begin{center}
$xR_{H}y$ si y sólo si $xy^{-1} \in H$\\
$xR^{H}y$ si y sólo si $x^{-1}y \in H$
\end{center}
\end{definition}

Tanto $R_{H}$ como  $R^{H}$ son relaciones de equivalencia.

\emph{Demostración: } Lo haremos para $R_{H}$ (para $R^{H}$ es análoga). Tenemos que ver que cumplen con la propiedad \textit{reflexiva} (1), \textit{simétrica} (2) y \textit{transitiva} (3)

(1). Si $x \in G$, $xx^{-1} = 1 \in H$ luego $xR_{H}x$.

(2). Si $xR_{H}y$ entonces $xy^{-1} \in H$, luego $$(xy^{-1})^{-1} \in H,$$ y esto es $yx^{-1} \in H$ así que $yR_{H}x$.

(3). Si $xR_{H}y$, y $yR_{H}z$, entonces, $$xy^{-1} \in H, \hspace{0.2cm} yz^{-1} \in H$$
y así $$xz^{-1} = (xy^{-1})(yz^{-1}) \in H$$ por lo que $xR_{H}z$.

$\hfill \square$

Notar que en las propiedades anteriores se ha tenido en cuenta, y esto resulta de gran importancia, que $H$ es subgrupo.

La demostración era bastante sencilla y casi evidente. El haber definido estas relaciones de equivalencia nos va a permitir estudiar las clases que éstas mismas generan para llegar a unos conjuntos especiales que llamaremos \textit{coclases} ó \textit{clases laterales}. En ocasiones se hace al revés, primero se presentan las coclases y a partir de ahí estudiamos (normalmente en sus demostraciones) las relaciones que definen.

Si $x \in G$, la clase de equivalencia de $x$ respecto de $R_H$ es $$Hx = \lbrace hx : h \in H \rbrace.$$ En efecto, $y \in G$ está relacionado con $x$ mediante $R_H$ si y sólo si $yx^{-1} = h \in H$, esto es, si $y = hx \in Hx$.

De igual modo, $y \in G$ está relacionado con $x$ mediante $R^H$ si y sólo si $x^{-1}y = h \in H$, o sea, si $y = xh \in xH$.

\begin{proposition}La aplicación entre los conjuntos cocientes $$
\begin{array}{rccl}
&G/R_H & \longrightarrow &G/R^H\\
&Hx& \longmapsto &x^{-1}H
\end{array}
$$ es biyectiva
\end{proposition}
\begin{proof}
Veamos primero que está bien definida. Si $Hx = Hy$ tenemos que $xR_Hy$, y así $xy^{-1} \in H$, luego $(x^{-1})^{-1}y^{-1} \in H$ y así $x^{-1}R^Hy^{-1}$, es decir, $x^{-1}H = y^{-1}H.$

La inyectividad se prueba de modo análogo: si $Hx \neq Hy$ entonces $xy^{-1} \notin  H$, luego $(x^{-1})^{-1}y^{-1} \notin H$, es decir, $x^{-1}$ e $y^{-1}$ no están relacionados mediante $R^H$, y por ello $x^{-1}H \neq y^{-1}H$.

Como cada clase $yH$ de $G/R^H$ es la imagen de $Hy^{-1}$ es claro que también es sobreyectiva.

\end{proof}

\begin{definition} \label{eq:indice} Decimos que $H$ es un subgrupo de $G$ de \textbf{índice infinito} si $G/R_{H}$ (y por ello $G/R^{H}$) es un conjunto infinito.


Cuando $G/R_{H}$ es finito, llamamos \textbf{índice} de $H$ en $G$, y lo denotamos $\left[ G:H \right]$, al número de elementos de $G/R_{H}$ (que además coincide con el de $G/R^{H}$). Es decir, definimos el índice como el número de coclases a derecha (o a izquierda porque es el mismo). En este caso decimos que $H$ es un subgrupo de $G$ de índice finito o que tiene índice finito en $G$. Por tanto tenemos que $$\left[ G:H \right] = card(G/R_{H}) = card(G/R^{H}).$$

Además es claro que si $G$ tiene orden finito, como la aplicación $$
\begin{array}{rccl}
&G & \longrightarrow & G/R_{H}\\
&x & \longmapsto &Hx
\end{array}
$$ es sobreyectiva, todo subgrupo de $G$ es de índice finito.
\end{definition}

Una consecuencia bastante clara de todo esto es que $[G:1] = o(G)$  $(=|G|)$ y $[G:H] = 1$ si y sólo si $G = H$.

\begin{example} Veamos cómo se relacionan los subgrupos de $\mathbb{Z}$ con el mismo $\mathbb{Z}$ a través de sus respectivos índices:

Sea $G = \mathbb{Z}$ y $\lbrace 0 \rbrace \neq H$ un subgrupo de $\mathbb{Z}$. Ya sabemos que $H$ es de la forma $H = m\mathbb{Z}$, con $m$ un entero positivo cualquiera. Como la operación en $\mathbb{Z}$ es la \textit{suma}, las clases respecto de $R_{H}$ serán de la forma $$H + x = m\mathbb{Z} + x, \hspace{0.1cm} x \in \mathbb{Z}.$$ Veamos que $$\mathbb{Z}/m\mathbb{Z} = \lbrace m\mathbb{Z} + 0, m\mathbb{Z} + 1, \ldots, m\mathbb{Z} + (m-1)\rbrace.$$ Dado $x \in \mathbb{Z}$ obtenemos, por el algoritmo de la división, $$x = qm + r, \hspace{0.1cm} 0\leq r\leq m-1.$$ Así $x-r = qm \in m\mathbb{Z} = H$, luego $xR_{H}r$, es decir, $m\mathbb{Z} + x = m\mathbb{Z} + r$, lo que prueba la igualdad. Además los elementos del segundo miembro son distintos, pues si $m\mathbb{Z} + k = m\mathbb{Z} + l, \hspace{0.1cm} 0\leq k < l \leq m-1$, entonces $lR_{H}k$, y por tanto $l-k \in m\mathbb{Z} = H, \hspace{0.1cm} 1\leq l-k <m,$ y tenemos que $l-k = qm$, con $q \in \mathbb{Z}$, lo cual implicaría que $l = qm +k > m$ si $q>0$ ó $k = l - qm >m$ si $q < 0$ ( y así $-q >0$), lo cual es imposible.

Así, $\left[ \mathbb{Z}:m\mathbb{Z} \right] = m$. Notar que $\mathbb{Z}$ es un grupo infinito cuyos subgrupos no nulos tienen índice finito.
\end{example}

$\hfill \blacksquare$

\begin{observation}\label{ob:preL} Si $G$ es un grupo, $H$ un subgrupo de $G$ y $x \in G$, las aplicaciones $$
\begin{array}{rccl}
&H & \longrightarrow &Hx\\
&h & \longmapsto &hx
\end{array}
$$
$$
\begin{array}{rccl}
&H & \longrightarrow &xH\\
&h & \longmapsto &xh
\end{array}
$$ son biyectivas. La inyectividad se deduce de las leyes de simplificación que vimos al principio, mientras que la sobreyectividad es obvia.
\end{observation}

\begin{observation}\label{ob:obNoBiy} De la observación anterior deducimos que, dado un $x \in G$, existe una biyección entre $Hx$ y $xH$. Sin embargo, $Hx$ y $xH$ pueden ser distintos (de hecho normalmente así será). Consideremos por ejemplo $G = D_n$ para algún $n \geq 3$, y con las notaciones vistas en~\ref{ex:gDie} $$H = \langle g \rangle, x = f.$$ Como $o(g) = 2$ tal y como vimos en~\ref{ej:gDie2}, $H = \lbrace 1,g \rbrace$, luego $Hf = \lbrace f, g \circ f \rbrace$, $fH = \lbrace f, f \circ g \rbrace$. Y en~\ref{prop:abDie} vimos que $f \circ g \neq g \circ f$, luego $Hf \neq fH$.
\end{observation}

\begin{theorem}[\textbf{\textit{Teorema de Lagrange}}]
Sean $G$ un grupo y $H$ un subgrupo de $G$. Son equivalentes:
\begin{enumerate}
\item $G$ es finito.
\item $o(H)$ es finito y $H$ tiene índice finito en $G$.
\end{enumerate}
En tal caso, $o(G) = o(H) \cdot [G:H]$. En particular, el orden de $H$ y el índice de $H$ en $G$ dividen al orden de $G$.
\end{theorem}
\begin{proof}
Veámoslo por doble implicación:

$1. \Rightarrow 2.$ Como $$\begin{array}{rccl}
&H & \longrightarrow & G\\
&x & \longmapsto &x
\end{array}$$ es inyectiva la finitud de $G$ implica la de $H$, y por~\ref{eq:indice} también lo es $G/R_{H}$.

$2. \Rightarrow 1.$ Como $R_{H}$ es relación de equivalencia, $G$ es unión \textit{disjunta} de las clases de equivalencia como ya sabemos. Así, recordamos que $$|G| = o(G) = \sum_{Hx\in G/R_{H}} card \hspace{0.1cm} (Hx).$$ Ahora, por~\ref{ob:preL}, $card \hspace{0.1cm} (Hx) = card \hspace{0.1cm} (H) = o(H)$, luego $$o(G) = o(H) \cdot card \hspace{0.1cm} G/R_{H} = o(H) \cdot [G: H],$$ y así $G$ es finito, y se tiene la conocida \textit{fórmula de Lagrange}.

\end{proof}

Como consecuencia inmediata se tiene que si $G$ grupo y $H$ subgrupo de $G$ son finitos, y es importante recalcar esto, entonces $$\left[ G:H \right] = \frac{|G|}{|H|}.$$

\begin{observation}Sean $G$ un grupo finito, $n = o(G)$ y $a \in G$. Entonces $a$ es elemento de torsión y $a^n = 1.$
\end{observation}
\begin{proof}
Como $\langle a \rangle \subseteq G$, $\langle a \rangle$ es finito, luego $a$ es de torsión. Si $m = o(a) = o(\langle a \rangle)$, el \textit{teorema de Lagrange} nos dice que $n = mp$, con $p \in \mathbb{N}$. Así, $a^n = a^{mp} = (a^m)^p = 1^p = 1.$

\end{proof}

Una consecuencia sencilla pero útil del \textit{teorema de Lagrange} es la siguiente:

\begin{corolario}Si $H$ y $K$ son subgrupos finitos de un grupo $G$ con $o(H) = m$, $o(K) = n$ y $mcd(m,n) = 1$, entonces $H\cap K = \lbrace 1_{G} \rbrace $.
\end{corolario}
\begin{proof}
$H\cap K$ es subgrupo de $H$ y de $K$, luego $o(H\cap K)$ debe dividir a $m$ y $n$. Como $mcd(m,n) = 1$, entonces $o(H \cap K) = 1$ y así $H \cap K = \lbrace 1_{G} \rbrace$.

\end{proof}

\begin{proposition}[\textbf{\textit{Transitividad del índice}}]\label{prop:tranIn} Sean $G$ un grupo y $H$ y $K$ subgrupos de $G$ tales que $H \subseteq K$. Entonces:
\begin{enumerate}
\item $H$ es subgrupo de $K$
\item Si el índice de $H$ en $G$ es finito lo son también el índice de $K$ en $G$ y el de $H$ en $K$, y $$\left[ G:H \right] = \left[ G:K \right]\cdot \left[ K:H \right].$$
\end{enumerate}
Esta propiedad se conoce como \textit{transitividad del índice.}
\end{proposition}
\begin{proof}
La primera afirmación es consecuencia obvia de las definiciones. Sea ahora $$
\begin{array}{rccl}
\pi \colon &G/R_H & \longrightarrow &G/R_K\\
&Hx & \longmapsto &Kx
\end{array}
$$
Está bien definida, ya que si $Hx = Hy$ entonces $xy^{-1} \in H \subseteq K$, y así $xR_Ky$ y tenemos $Kx = Ky$.

Como evidentemente es sobreyectiva, $$G/R_H = \bigcup_{Kx \in G/R_K}\pi^{-1}(Kx).$$ Además esta unión es claramente disjunta.

Notamos también por $R_H$ la restricción de $R_H$ a $K$. Nótese que la condición $Hy \in \pi^{-1}(Kx)$ equivale a decir que $Ky = Kx$, es decir, $z = yx^{-1} \in K$. De hecho, $$
\begin{array}{rccl}
&\pi^{-1}(Kx) & \longrightarrow &K/R_H\\
&Hy & \longmapsto &H(yx^{-1})
\end{array}
$$
es una biyección. 

La sobreyectividad de $\pi$ y la finitud de $G/R_H$ implican la de $G/R_K$, luego el índice de $K$ en $G$ es finito. 

Por otro lado, $\pi^{-1}(Kx) \subseteq G/R_H$ luego también es finito, y así, lo es el índice de $H$ en $K$. Finalmente, $$[G:H] = \sum_{Kx \in G/R_K} card \hspace{0.1cm}\pi^{-1}(Kx) = card\hspace{0.1cm} G/R_K \cdot card\hspace{0.1cm} K/R_H.$$ Por lo que $$[G:H] = [G:K] \cdot [K:H].$$

\end{proof}

\begin{proposition}Sean $G$ un grupo y $H,K$ subgrupos de $G$ de orden finito. Se tiene $$card \hspace{0.1cm} HK = \dfrac{o(H)\cdot o(K)}{o(H \cap K)}.$$
\end{proposition}
\begin{proof}
La relación $(h,k) R (h',k')$ si $hk = h'k'$ definida en $H \times K$ es, evidentemente, de equivalencia y la aplicación $$
\begin{array}{rccl}
&(H\times K)/R& \longrightarrow &HK\\
&[(h,k)]_R & \longmapsto &hk,
\end{array}
$$ donde $[(h,k)]_R$ denota la clase de $(h,k)$ respecto de $R$, es biyectiva ya que
\begin{enumerate}
\item Está bien definida, ya que si $[(h,k)]_R = [(h',k')]_R$ entonces $(h,k) R(h',k')$, es decir, $hk = h'k'$.
\item Es inyectiva, ya que $[(h,k)]_R \neq [(h',k')]_R$ quiere decir que $(h,k)$ y $(h',k')$ no están relacionados, luego $hk \neq h'k'$.
\item Es evidentemente sobreyectiva.
\end{enumerate}

Así, tenemos que $card \hspace{0.1cm} HK = card \hspace{0.1cm} (H \times K)/R$.

Como $$o(H) \cdot o(K) = card \hspace{0.1cm} (H \times K) = \sum_{[(h,k)]_R \in (H\times K)/R}card\hspace{0.1cm}[(h,k)]_R,$$ necesitamos calcular $card \hspace{0.1cm}[(h,k)]_R$.

Veamos que la aplicación $$
\begin{array}{rccl}
&[(h,k)]_R& \longrightarrow &H\cap K\\
&(u,v)& \longmapsto &u^{-1}h
\end{array}
$$
es una biyección.

Si $(u,v) \in [(h,k)]_R$ entonces $hk = uv$, luego $u^{-1}h = vk^{-1} \in H \cap K$ y la aplicación está bien definida. 

Es inyectiva, pues si $(u,v)$ y $(w,z)$ son elementos distintos en $[(h,k)]_R$, se tiene $$hk = uv = wz, \quad u \neq w~\acute{o}~ v \neq z.$$ 
Si $u \neq w$, $u^{-1}h \neq w^{-1}h$. Si $v \neq z$, $u^{-1}h = vk^{-1} \neq zk^{-1} = w^{-1}h$. Así, en cualquier caso $u^{-1}h \neq w^{-1}h$.

También es sobreyectiva ya que, dado $t \in H \cap K$, $(ht^{-1}, tk) \in [(h,k)]_R$ puesto que $ht^{-1} \in H$, $tk \in K$ y $(ht^{-1})(tk) = hk$, y se tiene $$(ht^{-1})^{-1}h = th^{-1}h = t.$$ Por lo tanto, $card \hspace{0.1cm}[(h,k)]_R = o(H \cap K)$, con lo que \begin{center}$o(H) \cdot o(K) = \sum_{[(h,k)]_R \in (H\times K)/R} o(H \cap K) = o(H\cap K) \cdot card \hspace{0.1cm} (H \times K)/R = o(H \cap K) \cdot card \hspace{0.1cm} (HK).$\end{center}

\end{proof}

\begin{observation}Sean $H_1, \ldots, H_t$ subgrupos de índice finito de un grupo $G$. Entonces $H = H_1 \cap \ldots \cap H_t$ es subgrupo de índice finito de $G$.
\end{observation}
\begin{proof}
Ya sabemos que $H$ es subgrupo de $G$. Además la aplicación $$\begin{array}{rccl}
&G/R_{H} & \longrightarrow & G/R_{H_{1}} \times \ldots \times G/R_{H_{t}}\\
&Ha & \longmapsto &(H_{1}a,\ldots, H_{t}a)
\end{array}$$
está bien definida y es inyectiva, pues si $Ha = Hb$, entonces $ab^{-1} \in H = H_{1}\cap \ldots \cap H_{t}$, y se tiene que $H_{j}a = H_{j}b$ para cada $1\leq j \leq t$. Por lo tanto, $$card(G/R_{H})\leq card(G/R_{H_{1}})\cdot \ldots \cdot card(G/R_{H_{t}}) = \left[ G:H_{1} \right] \cdot \ldots \cdot \left[ G:H_{t} \right]$$ y esto es finito.

\end{proof}

\begin{example} [\textbf{\textit{Subgrupos de $D_4$}}] \label{ex:subDie} Vamos a calcular todos los subgrupos del grupo diédrico $D_4$.

Como $o(D_4) = 8$, salvo los subgrupos triviales $\lbrace 1 \rbrace$ y $D_4$, todos los subgrupos de $D_4$ tienen, por el teorema de Lagrange, orden $2$ o $4$.

Si $H$ es subgrupo de orden dos será $H = \lbrace 1, h \rbrace$, con $h \in D_4$, $h \neq 1$. Y como $H$ es subgrupo, ha de ser $h \circ h \in H$, es decir, $h \circ h = h$ ó bien $h \circ h = 1$. Del primer caso deducimos que $h =1$, lo cual es falso. Así, $h \circ h = 1$, $h \neq 1$. Con las notaciones de~\ref{ex:gDie}, $h = f^i$ ó $h =g \circ f^i$, para algún $0 \leq i \leq 3.$ Recordemos que, tal y como vimos en~~, $o(f) = 4$ y $o(g)=2$. Si $h = f^i$, como $1 = h^2 =f^{2i}$, $2i$ ha de ser múltiplo de $4$, luego $i$ es par, con $0 \leq i \leq 3$. Así, $i = 0,2$. Para $i=0$, $h = f^0 = 1$ y no nos sirve. Para $i = 2$, obtenemos $h = f^2 \neq 1$, pues $o(f) = 4 > 2$, $h^2 = f^4 = 1$. Así, $H = \lbrace 1,f^2 \rbrace$ es subgrupo de orden dos.

Antes de calcular los demás subgrupos de orden dos necesitaremos: $$f^k \circ g \circ f^k = g,~para~cada~0 \leq k \leq n-1~en~D_4.$$
Veámoslo. Por inducción sobre $k$: si $k = 0$, es obvio. Para $k = 1$, si $V = \lbrace a_1, \ldots, a_n \rbrace$ son los vértices del polígono, \begin{center}$(f \circ g \circ f) (a_i) = f(g(a_{i+1})) = f(a_{n-(i+1)+2} = a_{n-(i+1)+2+1} = a_{n-i+2}=g(a_i)$ para $1 \leq i \leq n$ (llamando $a_{n+1} = a_1$). \end{center} Así, es claro que $f\circ g \circ f = g$. Ahora, si $k>1$, $$f^k \circ g \circ f^k = f\circ(f^{k-1} \circ g \circ f^{k-1}) \circ f= f \circ g \circ f = g,$$ usando la hipótesis de inducción y el caso $k = 1$. Entonces, para cada $0 \leq i \leq n-1$, $$(g \circ f^i)^2 = (g \circ f^i) \circ (g \circ f^i) = g \circ (f^i \circ g \circ f^i) = g \circ g = g^2 = 1,$$ y como ya vimos que $g \circ f^i \neq 1$ entonces si $H_i = \lbrace 1, g \circ f^i = h_i \rbrace$, $0 \leq i \leq n-1$, $o(h_i)=2$. En particular, $H = \lbrace 1, f^2 \rbrace$, $H_0, H_1, H_2,H_3$ son todos subgrupos de orden dos de $D_4$.

Sólo falta calcular los subgrupos de orden $4$. Sea $H$ uno de ellos. Supongamos que $f \in H$. Como $o(f) = 4$ y $ \langle f \rangle \subseteq H$, como $f \in H$ resulta $\lbrace 1,f,f^2,f^3 \rbrace = \langle f \rangle \subseteq H$ y $o(H) = 4$. Por lo tanto, si $f \in H$, ha de ser $H = \langle f \rbrace$, que evidentemente es un subgrupo de orden $4$.

Calculemos ahora los subgrupos de orden $4$ que no contienen a $f$. Para facilitar los cálculos observemos que si $H$ es un subgrupo de un grupo $G$, $x,y \in G$, $x\in H$, $xy \notin H$, entonces $y \notin H$, pues si $y \in H$, como $x \in H$ tendríamos $xy \in H$.

Sea pues $H$ un subgrupo de orden $4$ de $D_4$ que no contiene a $f$. Como $f^4 = 1$, entonces $f \circ f^3 = 1 = f^3f$, luego $f^3 = f^{-1} \notin H$. Supongamos ahora que $g \in H$. Como $g \circ (g \circ f) = f \notin H$, $g\in H$, se sigue que $g \circ f \notin H$. Si $f \circ g \in H$ entonces $f = (f \circ g) \circ g \in H$, lo cual es falso. Así, $f \circ g \notin H$ y como $f^3 \circ g \circ f^3 = g$ y $f^3 = f^{-1}$, se tiene que $f^{-1} \circ g \circ f^3 = g$, luego $g \circ f^3 = f \circ g \notin H$. Por lo tanto, $H \subseteq D_4 = \lbrace 1, f, f^2, f^3, g, g\circ f, g \circ f^2, g \circ f^3 \rbrace$, $o(H) = 4$, $f, f^3, g \circ f, g \circ f^3 \notin H$, luego $$H = \lbrace 1, f^2, g, g \circ f^2 \rbrace.$$

Comprobemos que esto es, efectivamente, un subgrupo de $D_4$. Notar que como $(f^2)^2 = g^2 = (g \circ f^2)^2 = 1$, se tiene que $(f^2)^{-1} = f^2$, $g^{-1} = g$, $(g \circ f^2)^{-1} = g\circ f^2$. Además, como $f^2 \circ g \circ f^2 = g$, es $g \circ f^2 = f^2 \circ g$, luego $$f^2 \circ g^{-1} = f^2 \circ g = g \circ f^2 \in H,$$ $$f^2 \circ (g\circ f^2)^{-1} = f^2 \circ g \circ f^2 = g \in H,$$ $$g \circ (f^2)^{-1} = g \circ f^2 \in H ,$$ $$g \circ (g \circ f^2)^{-1} = g \circ (g \circ f^2) = f^2 \in H,$$ $$(g \circ f^2) \circ (f^2)^{-1} = (g \circ f^2) \circ f^2 = g \in H,$$ $$(g \circ f^2) \circ g^{-1} = (g \circ f^2) \circ g = (f^2 \circ g) \circ g = f^2 \in H,$$ lo que prueba que $\lbrace 1, f^2, g, g \circ f^2 \rbrace$ es subgrupo de $D_4$.

Quedan por calcular los subgrupos $H$ de orden $4$ de $D_4$ que no contienen ni a $f$ ni a $g$. Por lo tanto, $f, f^3, g \notin H$. Si $f^2 \notin H$, tendríamos que $g \circ f, g \circ f^2 \in H$, luego su producto $(g \circ f) (g \circ f^2) = g \circ (f \circ g  \circ f) \circ f = g^2 \circ f = f \in H$, que es falso. Así, $f^2 \in H$ (luego ó $g\circ f$ ó $g \circ f^2 \notin H$). 

Si $g \circ f^2 \in H$, $g \circ f^2 \circ f^2 = g \in H$, que es falso. Así, $g \circ f^2 \notin H$, y necesariamente $$H = \lbrace 1, f^2, g\circ f, g \circ f^3 \rbrace.$$
Comprobemos que es un subgrupo de $D_4$. Notar que $(f^2)^2 = (g \circ f)^2 = (g \circ f^3)^2 = 1$, luego $(f^2)^{-1} = f^2$, $(g \circ f)^{-1} = g \circ f$, $(g \circ f^3)^{-1} = g \circ f^3$.

Ahora, tenemos $$f^2 \circ (g \circ f)^{-1} = f^2 \circ g \circ f = f \circ (f \circ g \circ f) = f \circ g = (f \circ g \circ f) \circ f^{-1} = g \circ f^{-1} = g \circ f^3 \in H,$$ $$f^2 \circ (g \circ f^3)^{-1} = f^2 \circ g \circ f^3 = (f^2 \circ g \circ f^2) \circ f = g \circ f \in H,$$ $$(g \circ f) \circ (f^2)^{-1} = g \circ f^3 \in H,$$ $$(g \circ f) \circ (g \circ f^3)^{-1} = (g \circ f)\circ(g \circ f^3) = g \circ (f \circ g \circ f)  \circ f^2 = g \circ g \circ f^2 = f^2 \in H,$$ $$(g \circ f^3) \circ (f^2)^{-1} = g \circ f^5 = g \circ f^4 \circ f = g \circ f \in H,$$ $$(g \circ f^3) \circ (g \circ f^{-1}) = g \circ f^3 \circ g \circ f = g \circ (f^3 \circ g \circ f^3) \circ f^2 = g \circ g \circ f^2 = f^2 \in H.$$

Resumiendo, además de $\lbrace 1 \rbrace$ y $D_4$, los subgrupos de $D_4$ son: 
\begin{enumerate}
\item $\lbrace 1, f^2 \rbrace$, $\lbrace 1,g \rbrace$, $\lbrace 1, g \circ f \rbrace$, $\lbrace 1, g \circ f^2 \rbrace$, $ \lbrace 1, g \circ f^3 \rbrace$, de orden $2$.
\item $\lbrace 1,f,f^2, f^3 \rbrace$, $\lbrace 1, f^2, g, g \circ f^2 \rbrace$, $\lbrace 1, f^2, g\circ f, g \circ f^3 \rbrace$, de orden $4$. 
\end{enumerate}
\end{example}

$\hfill \blacksquare$

\begin{example}[\textbf{\textit{El grupo cuaternión}}]

Consideremos los ocho símbolos siguientes: $$ Q = \lbrace 1,-1, i, j, k, -i,-j,-k \rbrace$$ y una operación $Q \times Q \longrightarrow Q$ que tiene a $1$ por elemento neutro, cumple la propiedad asociativa, la regla de los signos que todos conocemos (por ejemplo $i(-k) = -(ik)$) y \begin{equation*}
\begin{aligned}
&ij = k, \hspace{0.2cm} ji = -k\\
&jk = i, \hspace{0.2cm} kj = -i\\
&ki = j, \hspace{0.2cm} ik = -j\\
&i^{2} = j^{2} = k^{2} = -1
\end{aligned}
\end{equation*}

Con esto, está claro que $Q$ es un grupo de orden $8$. Sólo queda demostrar que tiene elemento inverso. 

Como se cumple la regla de los signos tenemos que $(-1)^{2} = 1$, luego $o(-1) =2$ y $-1$ es su propio inverso. Como $i^{2} = -1$, resulta que $(-i)^{4} = (-1)^{4}i^{4} = i^{4} = (-1)^{2} = 1,$ luego $o(i) = o(-i) = 4$, y así $i^{-1} = i^{3}$ ya que $ii^{3} = i^{4} = 1$, además $(-i)^{-1} = -i^{3}$ ya que $-i(-i)^{3} = (-i)^{4} = 1$. 

Análogamente, $o(j) = o(-j) = 4$, $o(k) = o(-k) = 4$ y $j^{-1} = j^{3}$, $k^{-1} = k^{3}$, $(-j)^{-1} = -j^{3}$, $(-k)^{-1} = -k^{3}$. Luego todos los elementos tienen inverso y así $Q$ es un grupo.


Veamos ahora cuáles son los subgrupos de $Q$. Evidentemente, $\lbrace 1 \rbrace$ y $Q$ lo son y por el \textit{Teorema de Lagrange} los demás han de tener orden $2$ ó $4$. Como $-1$ es el único elemento de orden $2$ de $Q$, $\lbrace 1, -1 \rbrace$ es el único subgrupo de orden $2$. 

Si $H$ es un subgrupo de orden $4$, deberá contener algún elemento $x$ que no sea el $1$ ó el $-1$. Entonces $\langle x \rangle \subseteq H$ y como $o(x) = 4=o(H)$ tendremos que $H = \langle x \rangle$. Además, como $-x = (-1)x = x^{2}x = x^{3} \in \langle x \rangle$ y $x  = (-1)(-x) = (-x)^{2}(-x) = (-x)^{3} \in \langle -x \rangle$, los subgrupos de orden $4$ de $Q$ serán $\langle i \rangle$, $\langle j \rangle$ y $\langle k \rangle$.


A este grupo $Q$ lo llamaremos \textbf{\textit{grupo cuaternión}}. Además estará generado por $i$ y $j$, es decir, $Q = \langle i,j \rangle$ ya que \begin{equation*}
\begin{aligned}
&i = i, \hspace{0.2cm} ij = k\\
&j = j, \hspace{0.2cm} i^{3}j = i^{2}ij =(-1)k = -k\\
&i^{0} = 1, \hspace{0.2cm} i^{3} = i^{2}i = (-1)i = -i\\
&i^{2} = -1, \hspace{0.2cm} i^{2}j = (-1)j = -j.
\end{aligned}
\end{equation*}
Y así, se tiene que $$Q = \lbrace 1, i, i^{2}, i^{3}, j, ij, i^{2}j, i^{3}j \rbrace.$$
Veremos enseguida que no existe un sistema generador con menos elementos. Para ello basta observar que $x^4 = 1$ para cada $x \in Q$, pues $$1^4 = 1,\hspace{0.3cm} (-1)^4 = ((-1)^2)^2 = 1^2 = 1, \hspace{0.3cm} i^4 = (i^2)^2 = (-1)^2 = 1$$ $$k^4 = (k^2)^2 = (-1)^2 = 1, \hspace{0.3cm} j^4 = (j^2)^2 = (-1)^2 = 1$$ $$(-i)^4 = i^4= 1, \hspace{0.3cm} (-j)^4 =j^4= 1,\hspace{0.3cm} (-k)^4 = k^4= 1$$ 

Este grupo además se suele presentar como el generado por las siguientes matrices: $$a = \left(
\begin{matrix}
i & 0 \\
0 & -i
\end{matrix}
\right) \hspace{0.3cm} b = \left(
\begin{matrix}
0 & 1 \\
-1 & 0
\end{matrix}
\right)$$

\end{example}

$\hfill \blacksquare$

\begin{definition}Un grupo $G$ se llama \textbf{cíclico} si existe un elemento $a \in G$ tal que $$G = \langle a \rangle.$$
\end{definition}

\begin{example}El grupo $\mathbb{Z}$ de los números enteros es un grupo cíclico pues $\mathbb{Z} = \langle 1 \rangle$.
\end{example}

$\hfill \blacksquare$

\begin{observation}Un grupo finito $G$ es cíclico si y sólo si existe $a \in G$ tal que $o(a) = o(G)$.
\end{observation}
\begin{proof}
En efecto, si $G = \langle a \rangle$, $o(G) = o(\langle a \rangle) = o(a)$. Recíprocamente, si $a \in G$ y $o(a) = o(G)$, $\langle a \rangle$ es un subconjunto de $G$ con tantos elementos como $G$, luego $$\langle a \rangle = G.$$

\end{proof}

\begin{observation}El grupo $D_4$ no es cíclico pues ningún elemento tiene orden $8$. De hecho, con las notaciones habituales, vimos en el ejemplo~\ref{ex:subDie} que $$o(f) = o(f^{-1}) = o(f^3) = 4,\hspace{0.2cm} o(1)=1,\hspace{0.2cm} o(x) = 2,~con~ x \in D_4 \setminus \lbrace 1, f, f^3 \rbrace.$$
Tampoco el grupo cuaternión es cíclico, puesto que $x^4= 1$ para cada $x \in Q$.

De este modo, cuando vimos que $D_4 = \langle f,g \rangle$, $Q = \langle i,j \rangle$ encontramos sistemas generadores de $D_4$ y $Q$ respectivamente con el menor número posible de elementos.
\end{observation}

\begin{proposition}Si $p$ es un número primo y $G$ es un grupo de orden $p$, $G$ es cíclico.
\end{proposition}
\begin{proof}
Sea $a\in G$, $a \neq 1$. Por el teorema de Lagrange $o(a)$ divide a $p$, como $o(a) \neq 1$, será $o(a) = p$, luego $G$ es cíclico.

\end{proof}

De hecho, se ha probado que $G = \langle a \rangle$ para cada $a \in G$, con $a \neq 1$.

\begin{proposition}Todo grupo cíclico es abeliano, pero existen grupos abelianos no cíclicos.
\end{proposition}
\begin{proof}
Para la primera parte, sea $G = \langle a \rangle$ un grupo cíclico. Dados $x,y \in G$, serán $x = a^k$, $y = a^l$, para ciertos $k,l$. Por lo tanto, $xy = a^ka^l = a^{k+l} = a^{l+k} = yx$ y así $G$ es abeliano.

Para la segunda parte, consideremos el subgrupo de $D_4$: $$H = \lbrace 1,f^2, g, g \circ f^2 \rbrace.$$ Como los elementos de $H$ tienen orden $2$, salvo $o(1) = 1$, y $o(H) = 4$, $H$ no es cíclico. Sin embargo, $H$ es abeliano: $$f^2 \circ g  = f^2 \circ g \circ f^4 = (f^2 \circ g \circ f^2) \circ f^2 = g \circ f^2,$$ $$f^2 \circ (g \circ f^2) = f^2 \circ g \circ f^2 = g = g \circ f^4 = (g \circ f^2) \circ f^2,$$ $$g \circ (g \circ f^2) = g^2 \circ f^2 = f^2 = (g \circ f^2) \circ (f^2 \circ g \circ f^2) = (g \circ f^2) \circ g.$$

\end{proof}

\begin{observation}En los ejemplos anteriores, \ref{ex:subDie} y~~, hemos visto que para los grupos $D_4$ y $Q$ se cumple una especie de recíproco al teorema de Lagrange:

Para cada divisor $m$ de $8 = o(D_4) = o(Q)$ existe un subgrupo de $D_4$ (respectivamente de $Q$) de orden $m$.

Este resultado, que como veremos más adelante es en general falso, se cumple para cualquier grupo cíclico finito, con una importante información adicional.
\end{observation}

\begin{proposition}Sea $G$ un grupo cíclico, $n = o(G)$. Para cada divisor $m$ de $n$ existe un único subgrupo de $G$ de orden $m$. Además este subgrupo es cíclico.
\end{proposition}
\begin{proof}
Sea $a \in G$ tal que $G = \langle a \rangle$. En primer lugar, 
si $n= kl$, $\langle a^{k} \rangle$ es un subgrupo de orden $l$, ya que $o(a^{k}) = \dfrac{n}{mcd(k,n)} = \dfrac{n}{k} = l$ por~\ref{prop:ordenes}.

Probemos la proposición. Como $m$ divide a $n$, existe un natural $d$ tal que $$n = dm.$$
Por lo que acabamos de ver al comienzo de la demostración, $H = \langle a^{d} \rangle$ tiene orden $m$. Veamos que es el único subgrupo de orden $m$. Sea $K$ otro subgrupo de $G$ de orden $m$. Sea $k$ el menor entero positivo tal que $a^{k} \in K$ (que existe puesto que $K \subseteq G = \langle a \rangle$).

Si $a^{p} \in K$, $p$ es múltiplo de $k$ ya que si dividimos $p$ entre $k$ tenemos, por el algoritmo de la división, que 
\begin{center}
$p = qk + r$, $0 \leq r < k$, luego $a^{r} = a^{p-qk} = a^{p}(a^{k})^{-q} \in K$
\end{center}
pero por la elección de $k$ (el menor entero positivo tal que $a^{k} \in K$), ha de ser necesariamente $r = 0$, y así $p= qk$, es decir, $p$ es múltiplo de $k$.

De esto se deduce que $n= sk$, con $s \in \mathbb{N}$, ya que $a^{n} = 1 \in K$, además $K = \langle a^{k} \rangle$ porque para cada $x = a^{p} \in K$ se tiene que $x = (a^{k})^{q} \in \langle a^{k} \rangle$.

Ahora, $m = o(K) = o(a^{k}) = n/k$, con lo que $k = n/m = d$ y así $K = \langle a^{d} \rangle = H$.

Luego, $\langle a^{d} \rangle$ es el único subgrupo de $G$ de orden $m$. Como además es cíclico, hemos acabado.

\end{proof}

\begin{proposition}Todo subgrupo de un grupo cíclico es cíclico.
\end{proposition}
\begin{proof}
Sea $G$ un grupo cíclico. Si $G$ es finito, ya hemos probado el resultado en la proposición anterior.

Supongamos así que $G$ no es finito, se razona igual. Si $G = \langle a \rangle$ y $H$ es subgrupo de $G$, consideramos $k$ el menor entero positivo tal que $a^k \in H$. Así, dado $x \in H \subseteq G$, será $x = a^p$, con $p \in \mathbb{N}$, y dividiendo $$p = qk + r, \quad 0 \leq r <k.$$ Como $a^r = a^p(a^k)^{-q} \in H$, debe ser $r = 0$ por la elección de $k$. Por lo tanto, $x = a^p = (a^k)^q \in \langle a^k \rangle$, con lo que $H = \langle a^k \rangle$.

\end{proof}

Terminamos este capítulo con un resultado que relaciona el orden de un grupo finito con el mínimo número de elementos de un sistema generador del grupo.

\begin{definition} Sea $G$ un grupo finitamente generado. Un sistema generador finito $S$ de $G$ se llama \textbf{minimal} si cualquier subconjunto de $G$ con menos elementos que $S$ no es sistema generador de $G$.
\end{definition}

Evidentemente, todo grupo finitamente generado tiene algún sistema generador minimal.

\begin{proposition}
Sea $G$ un grupo finito de orden $n$ y $S = \lbrace x_1, \ldots, x_p \rbrace$ un sistema generador minimal de $G$. Entonces $2^p \leq n$.
\end{proposition}
\begin{proof}Llamemos $S_i = \lbrace x_1, \ldots, x_i \rbrace$ para cada $1 \leq i \leq p$ y $H_i = \langle S_i \rangle$. Desde luego, $H_p = G$.

Evidentemente $H_i \subseteq H_{i+1}$ para cada $1 \leq i \leq p-1$, pues $H_{i+1} \supseteq S_{i+1} \supseteq S_i$ y $H_i$ es el menor subgrupo que contiene a $S_i$.

Además, el contenido es estricto; en caso contrario, $x_{i+1} \in H_{i+1} = H_i$, luego $x_{i+1} = x_i^{l_i} \ldots$ para ciertos enteros $l_1, \ldots, l_i$.

Consideremos $$T = \lbrace x_1, \ldots, x_i, x_{i+2}, \ldots, x_p \rbrace,$$ que tiene $p-1$ elementos. Si probamos que $T$ es sistema generador de $G$ habremos obtenido una contradicción, pues $S$ no sería minimal.

Dado $x \in G = \langle S \rangle$, se escribe: $$x = s_1^{h_1}\ldots s_m^{h_m}, \quad m \in \mathbb{N}, s_j \in \lbrace x_1, \ldots, x_p \rbrace, h_j \in \mathbb{Z}, 1 \leq j \leq m.$$

Cada vez que en la expresión anterior aparezca $s_j = x_{i+1}$ lo sustituimos por $x_{i+1} = x_i^{l_i}$. Así, $x \in \langle T \rangle$, y $T$ es sistema generador de $G$. 

Así (volviendo a lo que estábamos viendo), tenemos pues $H_1 \subsetneq H_2 \subsetneq \ldots \subsetneq H_p = G.$ Y aplicando reiteradamente la transitividad del índice~\ref{prop:tranIn} se deduce $$[G:H_1] = [H_p:H_{p-1}] \cdot [H_{p-1}:H_{p-2}] \cdot \ldots \cdot [H_2:H_1].$$ Usando el teorema de Lagrange, para cada $1 \leq i \leq p-1$, $[H_{i+1}:H_i] = \dfrac{o(H_{i+1})}{o(H_i)} >1$ pues $H_i \subsetneq H_{i+1}$, y como $[H_{i+1}:H_i]$ es un número entero, $[H_{i+1}:H_i] \geq 2$. Así, tenemos $$\dfrac{o(G)}{o(H_1)} = [G:H_1] \geq 2^{p-1},$$ y por lo tanto $n = o(G) \geq o(H_1) \cdot 2^{p-1}.$

Si $o(H_1) = 1$ sería $H_1 = \lbrace 1 \rbrace$, luego $x_1 = 1$ y $U = \lbrace x_2, \ldots, x_p \rbrace$ sería un sistema generador de $G$ con menos elementos que $S$. Esto es absurdo y por lo tanto $o(H_1) \geq 2$. En consecuencia, $n \geq 2 \cdot 2^{p-1} = 2^p$.

\end{proof}

\begin{observation}En determinadas situaciones, la cota $2^p \leq n$ es mejorable. Supongamos que $q$ es el menor número primo que divide a $n$. Como cada $[H_{i+1}:H_i] \neq 1$ tendremos $$[H_{i+1}:H_i]\geq q,$$ y análogamente $o(H_1)\geq q$. De este modo obtendremos $q^p \leq n$. Por ejemplo, si $n$ es impar tenemos $q \geq 3$ y así $3^p \leq n$.
\end{observation}
\section{Subgrupos normales. Grupos cocientes. Homomorfismos}

\subsection{Subgrupos normales. Propiedades}

Al trabajar con cualquier clase de objetos en Matemáticas es importante hallar relaciones de equivalencia tales que los cocientes admitan, de modo natural, una estructura del tipo de la de los objetos iniciales. En el caso de los grupos, si $H$ es un subgrupo de un grupo $G$, los cocientes $G/R_H$ y $G/R^H$ no admiten, en general, estructura de grupo de modo natural. Estudiaremos aquí los subgrupos $H$ para los que esto es posible. 

Posteriormente estudiaremos las aplicaciones entre grupos «compatibles» con la estructura de grupo. De la existencia de una tal aplicación entre dos grupos $G$ y $G'$ obtendremos información sobre $G'$ a partir de información sobre $G$ y recíprocamente.

\begin{proposition}\label{prop:gruNo} Sean $G$ un grupo y $H$ un subgrupo de $G$. Las siguientes condiciones son equivalentes:
\begin{enumerate}
\item Para cada $a \in G$, $Ha = aH$.
\item $H = H^{a}$ para cada $a \in G$. Es decir, $a^{-1}Ha = H \hspace{0.2cm} \forall a\in G$.
\item Para cada par de elementos $a,b \in G$ tales que $ab \in H$ se verifica que $ba \in H$.
\end{enumerate}
\end{proposition}
\begin{proof}Sigamos una demostración circular:

1 $\Rightarrow$ 2. Si $y \in H^{a}$ entonces $aya^{-1} = h \in H$. Como $ay = ha \in Ha = aH$ existirá un $h' \in H$ con $ay = ah'$. Simplificando tenemos que $y = h' \in H$, luego $H^{a}\subseteq H$. Y aplicando el contenido que acabamos de probar para $a^{-1}$ se tiene que $H^{a^{-1}}\subseteq H$, y así $H = (H^{a^{-1}})^{a}\subseteq H^{a}$, por lo tanto $H = H^{a}$.

2 $\Rightarrow$ 3. Como $ab \in H$ entonces $ba = a^{-1}(ab)a \in H^{a} = H.$

3 $\Rightarrow$ 1. Sea $x \in Ha$. Entonces, $x = ha$ con $h \in H$ y, por ello, $xa^{-1} = h \in H.$ Por hipótesis $a^{-1}x = h' \in H$, y así $x = ah'\in aH,$ demostrando el primer contenido $Ha \subseteq aH$.

Recíprocamente, si $x = ah \in aH$, resulta que $a^{-1}x = h \in H$, luego $xa^{-1} = h' \in H$, es decir, $x = h'a \in Ha,$ demostrando con esto el contenido recíproco $aH \subseteq Ha$, y por lo tanto $aH = Ha$.

\end{proof}

\begin{definition}Un subgrupo $H$ de un grupo $G$ que cumple cualquiera (y por tanto todas) de las condiciones de la propoposición anterior se llama \textbf{normal}. Con más precisión, diremos que $H$ es subgrupo normal de $G$.
\end{definition}

\begin{observation}Evidentemente la condición $(1)$ de~\ref{prop:gruNo} equivale a decir que $R_H = R^H$. En particular, si $H$ es normal, $G/R_H = G/R^H$ y denotaremos ambos cocientes por $G/H$. Más adelante veremos qué es exactamente este cociente.
\end{observation}

\begin{observation}\label{ob:FacNo} Para probar que $H$ es un subgrupo normal de $G$ basta ver que $H^a \subseteq H$ para cada $a \in G$. Probado esto y aplicado para $a^{-1}$, se tendrá $H^{a^{-1}} \subseteq H$, luego $H = (H^{a^{-1}})^a \subseteq H^a$ y por ello $H = H^a$ y así $H$ es normal.
\end{observation}

\begin{observation}\label{ob:abNo} Todo subgrupo de un grupo abeliano es normal, puesto que en un grupo abeliano siempre se cumple $ab = ba$, luego se tiene la condición $(3)$ de~\ref{prop:gruNo}.
\end{observation}

\begin{proposition}Si $G$ es un grupo y $H$ un subgrupo de $G$ con índice $2$, $H$ es subgrupo normal de $G$.
\end{proposition}
\begin{proof}
Por hipotesis, $G/R_H$ tiene dos elementos, y también $G/R^H$. 

Entonces, dado $a \in G$ puede ocurrir: 
\begin{enumerate}
\item $a \in H$. En tal caso, $a1^{-1} = a \in H$, luego $aR_H 1$ y $Ha = H1 = H$. Como $a^{-1}1 = a^{-1} \in H$, también se tiene $aR^H1$ y así $aH = 1H = H$. Luego, $aH = Ha$.
\item Si $a \notin H$, $aH \neq H$. Como $G/R^H$ tiene dos elementos, será $G = H \cup aH$ (unión disjunta). Así, $aH = G\setminus H = Ha$ y $H$ es normal.
\end{enumerate}

\end{proof}

\begin{observation}En cualquier grupo $G$ los subgrupos $\lbrace 1_G \rbrace$ y $G$ son normales. En efecto, dado $a \in G$ se tiene $a \lbrace 1_G \rbrace = \lbrace 1_G \rbrace a$ (de aquí se deduce que $G/\lbrace 1_G \rbrace = G$). Y $aG = G = Ga$ (ya que $ax = (axa^{-1})a$).
\end{observation}

\begin{observation}El recíproco de~\ref{ob:abNo} es falso. EL grupo cuaternión $Q$ no es abeliano, pero todos sus subgrupos son normales.
\end{observation}
\begin{proof}
Que no es abeliano es claro, pues $ij = k \neq -k = ji$. Los subgrupos $\lbrace 1 \rbrace$ y $Q$ son normales por lo que acabamos de ver. Los subgrupos de orden $4$ tienen índice $2$, luego son normales. El único subgrupo de orden $2$ es $H = \lbrace 1, -1 \rbrace$, que es normal porque para cada $a \in Q$, $$aH = a\lbrace 1, -1 \rbrace = \lbrace a, -a \rbrace = \lbrace 1, -1 \rbrace a = Ha.$$

\end{proof}

\begin{definition}Los grupos como $Q$, cuyos subgrupos son todos normales, se llaman \textbf{hamiltonianos}.
\end{definition}

\begin{observation}\label{ob:gruNNo} Existen grupos cuyos subgrupos no son todos normales. Consideremos $G = D_n$, con $n \geq 3$, y, con las notaciones usuales, $H = \lbrace 1, g\rbrace$. Vimos en~\ref{ob:obNoBiy} que $fH \neq Hf$, luego $H$ no es subgrupo normal de $D_n$.

Calculemos ahora los subgrupos normales de $D_4$. Evidentemente, $\lbrace 1 \rbrace$ y $D_4$ son normales.

Los subgrupos $\lbrace 1, f, f^2, f^3 \rbrace$, $\lbrace 1,f^2, g, g \circ f^2 \rbrace$ y $\lbrace 1,f^2, g \circ f, g \circ f^3 \rbrace$ son los subgrupos de $D_4$ de orden $4$. Como todos ellos tienen índice $2$, son normales.

Debemos decidir si los subgrupos de orden $2$, que son $$H_1 = \lbrace 1,f^2 \rbrace, \quad H_2 = \lbrace 1,g \rbrace, \quad H_3 = \lbrace 1, g \circ f \rbrace,$$ $$H_4 = \lbrace 1, g \circ f^2 \rbrace, \quad H_5 = \lbrace 1, g \circ f^3 \rbrace,$$ son normales.

\begin{enumerate}
\item $H_2$ y $H_4$.

Tenemos $f \circ g \circ f^{-1} = f \circ g \circ f^3 = (f \circ g \circ f) \circ f^2 = g \circ f^2 \in H_4$, luego $g \in H^f_4$. Como $H^f_4$ es un subgrupo de orden $2$ resulta que $H^f_4 = \lbrace 1,g \rbrace = H_2$. Así, $H_2^{f^{-1}} = (H^f_4)^{f^{-1}} = H_4$. En consecuencia, $H^f_4 = H_2 \neq H_4$, luego $H_4$ no es normal; $H^{f^{-1}}_2 = H_4 \neq H_2$, luego $H_2$ no es normal.
\item $H_3$ y $H_5$.

Tenemos $f \circ (g \circ f) \circ f^{-1} = (f \circ g \circ f) \circ f^3 = g \circ f^3 \in H_5$, luego $g \circ f \in H_5^f$ y al ser $H_5^f$ subgrupo de orden $2$ resulta que $H_5^f = \lbrace 1, g \circ f \rbrace = H_3$ y por lo tanto, $H_5 = H_3^{f^{-1}}$. Así, $H_5^f = H_3 \neq H_5$, luego $H_5$ no es normal; $H_3^{f^{-1}} = H_5 \neq H_3$, luego $H_3$ no es normal.

\item $H_1$.

Veamos, para acabar, que $H_1$ es subgrupo normal de $D_4$. En primer lugar, $a^{-1} \circ f^2 \circ a = f^2$ para cada $a \in D_4$, ya que si $a = f^i$ tenemos $a^{-1} \circ f^2 \circ f = f^{-i} \circ f^2 \circ f^i = f^{-i+2+i} = f^2$, y si $a = g \circ f^i$ tenemos $a^{-1} \circ f^2 \circ a = (g \circ f^i)^{-1} \circ f^2 \circ (g \circ f^i) = f^{-i} \circ g^{-1} \circ f^2 \circ g \circ f^i = f^{-i} \circ g^{-1} \circ (f^2 \circ g \circ f^2) \circ f^{i-2} = f^{-i} \circ g^{-1} \circ g \circ f^{i-2} = f^{-i} \circ f^{i-2} = f^{-2} = f^2.$

Ahora, dado $a \in D_4$, si $x \in H_1^a$ tendremos $axa^{-1} \in H_1$ y por lo tanto, $axa^{-1} = 1$ ó $axa^{-1} = f^2$. En el primer caso, $ax = a$, luego $x = 1 \in H_1$. En el segundo caso, $x = a^{-1}f^2a = f ^2 \in H_1.$ Así, $H_1^a \subseteq H_1$ para cada $a \in D_4$, con lo que, por~\ref{ob:FacNo}, $H_1$ es subgrupo normal de $D_4$. Por lo tanto, $H_1 = \lbrace 1, f^2 \rbrace$ es el único subgrupo normal de orden $2$ de $D_4$.
\end{enumerate}
\end{observation}

\begin{proposition} Si $G$ es un grupo, todo subgrupo $H \subseteq Z(G)$ es un subgrupo normal de $G$.
\end{proposition}
\begin{proof}
Recordemos que el centro de $G$ es $$Z(G) = \lbrace x \in G : ax = xa \hspace{0.2cm} \forall a \in G\rbrace$$ y es subgrupo de $G$.

Usando~\ref{ob:FacNo} bastará probar que $H^{a}\subseteq H$ para cada $a \in G$. Sea $x \in H^{a}$. Así $axa^{-1} = h \in H$, luego $x = a^{-1}ha$. Como $h \in H \subseteq Z(G)$, $ha = ah$ y así $$x = a^{-1}ha = h \in H.$$

\end{proof}

\begin{observation}[\textbf{\textit{Grupo especial lineal}}] Sea $n \in \mathbb{N}$ (no nulo) y $G = GL_n (\mathbb{R})$ el grupo de las matrices de orden $n$ con coeficientes en $\mathbb{R}$ y determinante no nulo. Ya sabemos que, con la operación producto de matrices, $G$ es un grupo.

Sea $H = \lbrace A \in G: det\hspace{0.1cm} A = 1 \rbrace.$ H es subgrupo normal de $G$, llamado \textbf{grupo especial lineal} y que se denota por $SL_n (\mathbb{R})$.

Dados $A, B \in H$, como $BB^{-1} = I_n$, $det\hspace{0.1cm} B \cdot det\hspace{0.1cm} B^{-1} = 1$, luego $det B^{-1} = \dfrac{1}{det\hspace{0.1cm} B} = 1$ y así $$det(AB^{-1}) = det\hspace{0.1cm} A \cdot det\hspace{0.1cm} B^{-1} = 1 \cdot 1 = 1.$$ En consecuencia, $AB^{-1} \in H$ y por lo tanto $H$ es subgrupo de $G$. Para probar que es normal veremos que se cumple la condición $(3)$ de~\ref{prop:gruNo}.

Si $A,B \in G$ y $AB \in H$ significa que $det(AB) = 1$. Entonces $det (BA) = det \hspace{0.1cm} B \cdot det A = det\hspace{0.1cm} A \cdot det\hspace{0.1cm} B = det(AB) = 1$, es decir, $BA \in H$.

\end{observation}

\begin{proposition}Sean $G$ un grupo y $H$ un subgrupo de $G$.
\begin{enumerate}
\item $H$ es subgrupo de $N_G(H)$.
\item $H$ es subgrupo normal de $N_G(H)$.
\item Si $K$ es un subgrupo de $G$ que contiene a $H$ y $H$ es un subgrupo normal de $K$, entonces $K \subseteq N_G(H)$.
\end{enumerate}
\end{proposition}
\begin{proof}Veámoslo por partes.
\begin{enumerate}
\item Basta ver que $H \subseteq N_{G}(H)$, o lo que es lo mismo, que si $a \in H$ entonces $H = H^{a}$.

Pero si $x \in H^{a}$ es $axa^{-1} = h \in H$, luego $x = a^{-1}ha \in H$. Por lo tanto $H^{a} \subseteq H.$ Recíprocamente, como también $a^{-1} \in H$ tendremos por lo anterior $H^{a^{-1}} \subseteq H$ y así $H = (H^{a^{-1}})^{a} \subseteq H^{a}$ y de aquí $H = H^{a}$.
\item Para cada $a \in N_{G}(H)$ es $H = H^{a}$ por definición. Así que $H$ es subgrupo normal de $N_{G}(H)$.
\item Si $a \in K$, como $H$ es subgrupo normal de $K$, es $H = H^{a}$, con lo que $a \in N_{G}(H)$.
\end{enumerate}

\end{proof}

\begin{proposition}Si $H$ y $K$ son subgrupos de un grupo $G$ decimos que $K$ es un \textbf{subgrupo conjugado} de $H$ si existe $a \in G$ tal que $$K = H^a.$$
\begin{enumerate}
\item Si $K$ es conjugado de $H$, entonces $H$ es conjugado de $K$, y diremos que $H$ y $K$ son \textit{conjugados}.
\item Si $\Sigma$ es la familia de subgrupos conjugados de $H$ (distintos) y $N = N_{G}(H)$ es el normalizador de $H$ en $G$, la aplicación $$
\begin{array}{rccl}
\varphi \colon &G/R_{N} & \longrightarrow & \Sigma\\
&Na & \longmapsto &H^{a}
\end{array}
$$ es biyectiva.
\item En particular, si $N_{G}(H)$ tiene índice finito en $G$, el número de conjugados de $H$ en $G$ es $\left[ G:N_{G}(H) \right]$.
\end{enumerate}
\end{proposition}
\begin{proof}Veámoslo por partes.
\begin{enumerate}
\item Es evidente, pues si $K = H^{a}$, $K^{a^{-1}} =(H^{a})^{a^{-1}} = H$.
\item Comencemos por demostrar que $\varphi$ está bien definida:

Si $Na = Nb$, entonces $ab^{-1} \in N$, luego $H^{ab^{-1}} = H$ y así $$H^{b} = (H^{ab^{-1}})^{b} = H^{a},$$ Veamos ahora que es inyectiva: 

Si $H^{a} = H^{b}$ se tiene $H^{ab^{-1}} = (H^{b})^{b^{-1}} = H$, luego $ab^{-1} \in N$ y $Na = Nb.$ Como la sobreyectividad es evidente, queda demostrado.
\item  Es claro ya que $$card\hspace{0.1cm} \Sigma = card(G/R_{N}) = \left[ G:N \right].$$
\end{enumerate}

\end{proof}

\begin{proposition}\label{eq:nhnormal} Sea $N$ un subgrupo normal de un grupo $G$ y sean $H$ y $K$ subgrupos de $G$ tales que $H$ es subgrupo normal de $K$. Entonces $NH$ es subgrupo normal de $NK$.
\end{proposition}
\begin{proof}
Primeramente veamos que $NH=HN$ y así $NH$ es subgrupo de $G$:

En particular $NH$ es subgrupo de $NK$, pues $NH \subseteq NK$. Pero si $x \in NH$ se escribirá $x = nh, \hspace{0.1cm} n \in N, \hspace{0.1cm} h \in H$. Así $x \in Nh = hN \subseteq HN$, la igualdad $Nh = hN$ se tiene por ser $N$ subgrupo normal de $G$. Esto prueba el contenido $NH \subseteq HN$. El otro es análogo. De igual forma se prueba que $NK = KN$, luego $NK$ es subgrupo de $G$, y así es grupo. Ahora veamos la normalidad:

Usaremos la condición $(1)$ de~\ref{prop:gruNo}. Veamos que si $a \in NK$, entonces $a(NH) = (NH)a$. Como $a \in NK$ se escribirá $a = nk, \hspace{0.1cm} n \in N, \hspace{0.1cm} k \in K$. Si $x \in a(NH) = a(HN)$ se tendrá $x = ahn_{1}, \hspace{0.1cm} h \in H, \hspace{0.1cm} n_{1} \in N.$ Como $x \in (ah)N = N(ah)$ por ser $N$ subgrupo normal de $G$, tendremos entonces $x = n_{2}ah = n_{2}nkh, \hspace{0.1cm} n_{2} \in N.$ Como $kh \in kH = Hk$ por ser $H$ subgrupo normal de $K$, $x = n_{2}nh_{1}k, \hspace{0.1cm} h_{1} \in H,$ o también, $x = n_{2}nh_{1}n^{-1}nk = n_{2}nh_{1}n^{-1}a$. Ahora $h_{1}n^{-1} \in HN = NH$, con lo que se tiene $h_{1}n^{-1} = n_{3}h_{2}$, $n_{3} \in N$, $h_{2} \in H$. Finalmente, $x = n_{2}nn_{3}h_{2}a \in (NH)a$. Y así $a(NH) \subseteq (NH)a$. Para el otro contenido se procede de igual forma.

\end{proof}

\begin{definition}Decimos que un grupo $G$ es simple si sus únicos subgrupos normales son $\lbrace 1_G \rbrace$ y $G$. Los ejemplos más sencillos de grupos simples son los de orden primo. De hecho
\end{definition}

\begin{observation}Si $p$ es un número primo y $G$ un grupo de orden $p$, los únicos subgrupos de $G$ son $\lbrace 1_{G}\rbrace$ y $G$. En particular, $G$ es simple.

Esto se sigue del hecho que si $H$ es un subgrupo de $G$, su orden debe dividir a $p$ por el teorema de Lagrange. Y como $p$ es primo, ó bien $o(H) = 1$ y así $H = \lbrace 1_{G}\rbrace$, ó bien $o(H) = p$ y así $H = G$.
\end{observation}

\begin{observation}La normalidad no es una propiedad \textbf{transitiva}, esto es, existen un grupo $G$ y subgrupos suyos $H$ y $K$ con $H \subseteq K$, $H$ subgrupo normal de $K$, $K$ subgrupo normal de $G$, pero $H$ no es subgrupo normal de $G$.

Por ejemplo, tomemos $G = D_4$ y con las notaciones usuales, $$K = \lbrace 1, f^2, g, g \circ f^2 \rbrace, \quad H = \lbrace 1,g \rbrace.$$ Como $[D_4 : K ] = 2$, $K$ es subgrupo normal de $D_4$. Como $[K:H] = \dfrac{o(K)}{o(H)} = 2$, también $H$ es subgrupo normal de $K$. Sin embargo, ya vimos en~\ref{ob:gruNNo} que $H$ no es subgrupo normal de $G$.
\end{observation}

\begin{definition}Si $H$ es un subgrupo de un grupo $G$, se llama \textbf{corazón} de $H$ a $$K(H) = \bigcap_{a\in G} H^a.$$ Por~\ref{ob:intGru}, $K(H)$ es subgrupo de $G$. Además, $K(H)$ es subgrupo normal de $G$.
\end{definition}
\begin{proof}
Basta probar que $K(H)^{b} \subseteq K(H)$ para cada $b \in G$:

Sea $x \in K(H)^{b}$, tenemos que ver que $x \in H^{a}$ para cada $a \in G$. Pero $
bxb^{-1} \in K(H) \subseteq H^{ab^{-1}}$, luego $$ab^{-1}(bxb^{-1})(ab^{-1})^{-1} \in H,$$ y por lo tanto $axa^{-1} \in H$, esto es, $x\in H^{a}$.

\end{proof}

Veamos por último que 

\begin{observation}Si $N \subseteq H$ es un subgrupo normal de $G$, entonces $N \subseteq K(H)$, puesto que, para cada $a \in G$, $$N = N^a \subseteq H^a,~luego~N \subseteq \bigcap_{a\in G} H^a = K(H).$$
\end{observation}

Ahora podemos demostrar:

\begin{theorem}[\textbf{\textit{Teorema de Poincaré}}]
Si $G$ posee un subgrupo de índice finito, también posee un subgrupo normal de índice finito.
\end{theorem}
\begin{proof}
Probemos que si $H$ es un subgrupo de índice finito, $K(H)$, que es normal, tiene índice finito.

Como $\left[ G:H \right]$ es finito, también lo es $$\left[ G:N_{G}(H) \right] = \frac{\left[ G:H \right]}{\left[ N_{G}(H):H \right]},$$ luego sabemos que $H$ tiene un número finito de conjugados. Y como $K(H)$ es la intersección de los conjugados de $H$, y hay una cantidad finita de éstos, para probar que $\left[ G:K(H) \right]$ es finito basta (ya que la intersección de subgrupos de índice finito es subgrupo de índice finito) demostrar que cada $H^{a}$ es subgrupo de $G$ de índice finito. De hecho probaremos la igualdad $$\left[ G:H \right] = \left[ G:H^{a} \right].$$ Para eso es suficiente demostrar que la aplicación $$
\begin{array}{rccl}
&G/R_{H^{a}} & \longrightarrow & G/R_{H}\\
&H^{a}x & \longmapsto &Hax
\end{array}
$$ es biyectiva.


Está bien definida, y es inyectiva, pues $H^{a}x = H^{a}y$ equivale a que $xy^{-1} \in H^{a},$ y así $axy^{-1}a^{-1} \in H,$ o lo que es lo mismo, $ax(ay)^{-1} \in H$ y esto es $Hax = Hay.$ Además es sobreyectiva, puesto que $Hy = Hax$ con $x=a^{-1}y$ para cada $y \in G.$

\end{proof}

Como adelantamos en la introducción, buscamos subgrupos $H$ tales que $G/R_H$ tenga, de modo natural, estructura de grupo. La siguiente proposición pone de manifiesto que los subgrupos normales son los adecuados.

\subsection{Grupos cocientes}

\begin{proposition}\label{eq:cociente} Sean $G$ un grupo y $H$ un subgrupo normal de $G$. El cociente $G/H = G/R_{H} = G/R^{H}$ tiene estructura de grupo con la operación $$\begin{array}{rccl}
&G/H \times G/H & \longrightarrow & G/H\\
&(aH,aH) & \longmapsto &abH.
\end{array}
$$
El elemento neutro es $H=1H.$ Además, si $H$ es subgrupo de $G$ de índice finito, $o(G/H) = \left[ G:H \right]$.
\end{proposition}
\begin{proof}
Ya sabemos que cuando $H$ es normal, los conjuntos cocientes $G/R_{H}$ y $G/R^{H}$ coinciden, y los denotaremos por $G/H$. El único punto problemático, y donde se hace uso de la normalidad de $H$, es cuando hay que comprobar que la operación está bien definida, es decir, que no dependa de los representantes $a$ y $b$ elegidos.
\begin{enumerate}
\item Sea pues $aH = xH$, $bH = yH$, comprobemos que $abH =xyH,$ es decir que $(ab)^{-1}xy \in H$, y así $b^{-1}a^{-1}xy \in H.$
Como $aH = xH$, $a^{-1}x = h \in H$. Como $bH = yH$, $b^{-1}y = h' \in H$. Por ello, $$b^{-1}a^{-1}xy = b^{-1}hy = b^{-1}yy^{-1}hy = h'y^{-1}hy.$$
Si $z = y^{-1}hy$, resulta que  $z \in H^{y} = H$ por ser $H$ normal. Por lo que, $$b^{-1}a^{-1}xy = h'z \in H.$$
\item Además la operación es asociativa, pues \begin{center}$aH((bH)(cH)) = (aH)(bcH) = (a(bc))H = ((ab)c)H = ((ab)H)(cH) = ((aH)(bH))cH.$\end{center}
\item Como
\begin{center}
$(aH)H=(aH)(1H)=(a1)H=(aH)$ y\\
$H(aH)=(1H)(aH)=(1a)H=aH$,
\end{center}
la clase $H$ es el elemento neutro. 
\item Dado $aH \in G/H$ se verifica $$(aH)(a^{-1}H) = (aa^{-1}H) = 1H = H,$$ $$(a^{-1}H)(aH) = (a^{-1}aH) = 1H = H,$$ y así $a^{-1}H$ es el inverso de $aH$. Es decir $$(aH)^{-1} = a^{-1}H.$$
\item Finalmente, si $H$ tiene índice finito en $G$, $$|G/H| = card(G/R_{H}) = \left[ G:H \right].$$
\end{enumerate}

\end{proof}


Es decir, si tenemos un grupo $G$ y un subgrupo normal $H$, entonces el cociente $G/H$ es también un grupo. Los subgrupos normales son los adecuados para dotar a un cociente de estructura de grupo. 

\begin{observation}Vamos a estudiar cómo son los subgrupos de un grupo cociente $G/H$.

Si $K$ es un subgrupo de $G$ que contiene a $H$, $H$ es subgrupo normal de $K$, por serlo de $G$. Entonces tiene sentido considerar el grupo cociente $K/H$. Evidentemente $K/H \subseteq G/H$ y es subgrupo de $G/H$, puesto que dados $aH, bH \in K/H$, $a,b \in K$ se tiene $$(aH)(bH)^{-1} = (aH)(b^{-1}H) = ab^{-1}H \in K/H,$$ ya que al ser $K$ subgrupo de $G$, $ab^{-1} \in K$.

Recíprocamente, sea $M$ un subgrupo de $G/H$, y llamemos $$K = \lbrace x \in G : xH \in M \rbrace.$$ Veamos que $K$ es un subgrupo de $G$ que contiene a $H$ y que $M = K/H$.

Desde luego, si $h \in H$ se tiene que $hH = H$ que pertenece a $M$ pues $M$ es subgrupo y $H$ es el neutro de $G/H$. Esto prueba que $h \in K$ y con ello que $H \subseteq K$.

Dados $x,y \in K$, tenemos $xH, yH \in M$ de donde $$xy^{-1}H = (xH)(y^{-1}H) = (xH)(yH)^{-1} \in M$$ por ser $M$ subgrupo. Esto quiere decir que $xy^{-1} \in K$. Por lo tanto $K$ es subgrupo de $G$.

Veamos que se cumple la igualdad $M = K/H$. Dado $xH \in K/H$, es $x \in K$ y por tanto $xH \in M$. En el otro sentido, si $xH \in M$, entonces $x \in K$, luego $xH \in K/H$. Es inmediato que si $K_{1}$ y $K_{2}$ son subgrupos de $G$ que contienen a $H$ y $K_{1}/H = K_{2}/H$, entonces $K_{1} = K_{2}$.
\end{observation}

Por lo tanto, hemos demostrado que la aplicación $$\begin{array}{rccl}
&K & \longrightarrow & K/H\\
&x & \longmapsto &xH.
\end{array}
$$ es una biyección entre los subgrupos de $G$ que contienen a $H$ y los subgrupos de $G/H$. Este resultado se conoce como \textbf{\textit{Teorema de la correspondencia}}. Además la biyección preserva la normalidad y por lo tanto tenemos que:

\begin{proposition} K es subgrupo normal de $G$ si y sólo si $K/H$ es subgrupo normal de $G/H$.\end{proposition}
\begin{proof}
Utilizaremos la condición $(3)$ de~\ref{prop:gruNo}. Supongamos que $K$ es normal. Dados $aH, bH$ con $(aH)(bH) \in K/H$, entonces $(ab)H \in K/H$, es decir, $ab \in K$.
Como $K$ es normal, y $ab \in K$, deducimos que $ba \in K$, luego $(bH)(aH) = (ba)H \in K/H,$ y así $K/H$ es normal. Para el recíproco es análogo.

\end{proof}

\begin{observation}Si $G$ es cíclico y $H$ es un subgrupo normal de $G$, también el cociente $G/H$ es cíclico. 

En efecto, si $G = \langle a \rangle$, es obvio que $G/H = \langle aH \rangle$.
\end{observation}

\begin{example}Dado un entero positivo $m$, el subgrupo $H = m\mathbb{Z}$ del grupo $\mathbb{Z}$ es desde luego normal, por ser $\mathbb{Z}$ abeliano. Como aquí la notación es aditiva, la operación en el cociente vendrá dada por 
$$\begin{array}{rccl}
&\mathbb{Z}/m\mathbb{Z} \times \mathbb{Z}/m\mathbb{Z} & \longrightarrow & \mathbb{Z}/m\mathbb{Z}\\
&(a+m\mathbb{Z}, b+m\mathbb{Z}) & \longmapsto &a+b+m\mathbb{Z}.
\end{array}
$$

Además el grupo cociente $\mathbb{Z}/m\mathbb{Z}$ es cíclico de orden $m$. Y sabemos que $$\mathbb{Z}/m\mathbb{Z} = \lbrace 0+m\mathbb{Z}, 1+m\mathbb{Z}, \ldots, (m-1)+m\mathbb{Z}\rbrace,$$ siendo todos los elementos del miembro de la derecha distintos. Visto esto tendremos $$o(\mathbb{Z}/m\mathbb{Z}) = m, \quad \mathbb{Z}/m\mathbb{Z} = \langle 1+m\mathbb{Z}\rangle.$$
\end{example}

$\hfill \blacksquare$

\begin{example}Sea $m$ un entero positivo. Denotamos 

$$\mathbb{Z}_{m}^{\ast} = \lbrace a +m\mathbb{Z} \in \mathbb{Z} / m\mathbb{Z} : mcd(a,m)= 1 \rbrace,$$
y consideremos la operación binaria $$\begin{array}{rccl}
&\mathbb{Z}_{m}^{\ast} \times \mathbb{Z}_{m}^{\ast} & \longrightarrow & \mathbb{Z}_{m}^{\ast}\\
&(a+m\mathbb{Z},b+m\mathbb{Z})& \longmapsto &ab+m\mathbb{Z}.
\end{array}
$$
Queremos demostrar que, con esta operación, $\mathbb{Z}^\ast_m$ es un grupo abeliano. En primer lugar hemos de ver que la operación está bien definida, y para ello 
\begin{enumerate}
\item Si $a + m\mathbb{Z} = a' + m\mathbb{Z}$ y $b + m\mathbb{Z} = b' + m\mathbb{Z}$, tendremos que $a = a' + mu$, $b = b' + mv$, con $u,v \in \mathbb{Z}$ y así $$ab = a'b' + m(b'u + a'v + muv).$$ Luego $ab - a'b' \in m\mathbb{Z}$ y por tanto $ab + m\mathbb{Z} = a'b' + m\mathbb{Z}.$ Esto demuestra que la definición no depende de los representantes.
\item Ahora veamos que es interna:

Si $mcd(a,m) = mcd(b,m) = 1$, entonces $mcd(ab,m) = 1$. Usando~\ref{ex:idBez} (Identidad de Bézout) tenemos que $$1 = ua +vm, \hspace{0.1cm} 1= u'b + v'm, \hspace{0.1cm} u,v,u',v' \in \mathbb{Z}.$$ Por lo que, $$1 = (ua +vm)(u'b + v'm)= uu'ab+(auv'+bvu' + mvv')m = u'' ab + v''m,$$
con $u''= uu'$ y $v''= auv'+bvu' + mvv'$. Y de nuevo, por~\ref{ex:idBez}, $mcd(ab,m)=1$ como queríamos ver.

El resto de las propiedades de grupo son fáciles de comprobar. La asociatividad es obvia, y es claro que $1 + m\mathbb{Z}$ es el elemento neutro. También es inmediato que $\mathbb{Z}^\ast_m$ es abeliano. 

Por último, para cada $a + m\mathbb{Z}$, como $mcd(a,m)=1$, se tiene que $1= au+mv$, $u,v \in \mathbb{Z}$, y así $1 + m\mathbb{Z} = (au+ m\mathbb{Z}) + (mv + m\mathbb{Z}) = (a + m\mathbb{Z})(u+ m\mathbb{Z}),$ por lo que $u + m\mathbb{Z}$ es el inverso de $a + m\mathbb{Z}$.

A la función $$\begin{array}{rccl}
\phi\colon &\mathbb{N}\setminus \lbrace 0 \rbrace & \longrightarrow &\mathbb{N}\setminus \lbrace 0 \rbrace\\
&m& \longmapsto &\phi (m)
\end{array}
$$ que a cada natural positivo $m$ le hace corresponder el orden $\phi(m)$ del grupo $\mathbb{Z}_{m}^{\ast}$ se le llama \textbf{función de Euler}.

Vamos a dar un procedimiento para calcular $\phi(m)$.

\begin{itemize}
\item Si $p$ es primo, $\phi(p) = p-1$, pues cada natural $1 \leq a \leq p-1$ cumple $mcd(a,p)=1.$
\item Si $p$ es un número primo y $m$ un natural positivo, $$\phi(p^{m})= p^{m-1}(p-1).$$ Esto se desprende del hecho de que los naturales $1 \leq a \leq p^{m}$ que no son primos con $p^{m}$ son $$p, 2p, 3p, \ldots, p^{m-1}p.$$ Así que hay $p^{m-1}$ que no son primos con $p$, por lo que $$\phi(p^{m}) = p^{m}-p^{m-1} = p^{m-1}(p-1).$$
\item Si $m$ y $n$ son primos primos entre sí, $\phi(mn)= \phi(m) \phi(n)$. Para ver esto se trata de encontrar una biyección 
$$\begin{array}{rccl}
&\mathbb{Z}_{mn}^{\ast} & \longrightarrow &\mathbb{Z}_{m}^{\ast} \times \mathbb{Z}_{n}^{\ast},
\end{array}
$$ 
ya que el primer miembro tiene $\phi(mn)$ elementos, y el segundo $\phi(m)\phi(n)$. Y, ¿cúal es esta biyección?, bien pues aquí la tenemos 
$$\begin{array}{rccl}
f \colon &a + mn\mathbb{Z}& \longrightarrow &(a+ m\mathbb{Z}, a + n\mathbb{Z}).
\end{array}
$$ 
Es claro que, si $mcd(a,mn)=1$, se tendrá $$mcd(a,m)= mcd(a,n)=1.$$ Además, si $a \in mn\mathbb{Z}$, también $a \in m\mathbb{Z}$ y $a \in n\mathbb{Z}$. Esto prueba que $f$ está bien definida.

También es inyectiva: si $a + m\mathbb{Z} = b + m\mathbb{Z}$ y $a + n\mathbb{Z} = b + n\mathbb{Z}$, resulta que $a-b$ es múltiplo de $m$ y de $n$. Y como $m$ y $n$ son primos entre sí, deducimos que $a-b$ es múltiplo de $mn$, luego $a + mn\mathbb{Z}= b + mn\mathbb{Z}$.

Lo más sorprendente es que es sobreyectiva. Sea $(x + m\mathbb{Z}, y + n\mathbb{Z}) \in \mathbb{Z}_{m}^{\ast} \times \mathbb{Z}_{n}^{\ast}$. Como $m$ y $n$ son primos entre sí, por~\ref{ex:idBez} existen $u,v \in \mathbb{Z}$ tales que $um + vn = 1$. Sea entonces $$a = yum + xvn.$$
Veamos que si $a + mn\mathbb{Z} \in \mathbb{Z}_{mn}^{\ast}$, entonces $f(a + mn\mathbb{Z}) = (x + m\mathbb{Z}, y + n\mathbb{Z})$ probando así la sobreyectividad:

Si $mcd(a,mn) \neq 1$, existirá un primo $p$ que dividirá a ambos. Como $p$ es primo y divide a $mn$, divide a uno de ellos, digamos $m$. Entonces también divide a $$a - yum = xvn,$$ y por ello $p$ ha de dividir a $x, v$ o $n$. Como $mcd(m,x) = mcd(m,n)=1$, se sigue que $p$ divide a $v$. En tal caso dividirá a $$um + vn = 1,$$ lo cual es absurdo. Por lo tanto, $a + mn\mathbb{Z} \in \mathbb{Z}_{mn}^{\ast}$. Como $$a - x = yum + x(vn- 1) = yum - xum \in m\mathbb{Z},$$ $$a - y = xvn + y(um-1) = xvn -yvn \in n\mathbb{Z},$$ y así tenemos que $$f(a + mn\mathbb{Z}) = (a + m\mathbb{Z}, a + n\mathbb{Z}) = (x + m\mathbb{Z}, y + n\mathbb{Z}).$$
\item Por fin, descomponiendo en factores primos $m = p_{1}^{a_{1}} \ldots p_{k}^{a_{k}}$, resulta $$\phi(m) = \phi(p_{1}^{a_{1}} \ldots p_{k}^{a_{k}}) = p_{1}^{a_{1}-1} \ldots p_{k}^{a_{k}-1}(p_{1}-1) \ldots (p_{k}-1).$$
\end{itemize}

\end{enumerate}
\end{example}

$\hfill \blacksquare$

Ya sabemos que los grupos $\mathbb{Z}$ y $\mathbb{Z}/m\mathbb{Z}$, con $m$ un entero positivo, son cíclicos. De hecho, éstos son los únicos grupos cíclicos. Precisaremos enseguida lo que significa esto.

\subsection{Homomorfismos}

\begin{definition}Una aplicación $f\colon G \longrightarrow  G'$ entre dos grupos $G$ y $G'$ se llama \textbf{homomorfismo de grupos} si
\begin{center}
$f(ab) = f(a)f(b)$ para cada $a,b \in G$.
\end{center}
\end{definition}

\begin{observation} Algunas propiedades sobre los homomorfismos de grupos que serán importantes tenerlas en cuenta: (consideraremos $f$ un homomorfismo cualquiera)
\begin{enumerate}
\item $f(1_{G}) = 1_{G'}$ ya que $$1_{G'}f(1_{G}) = f(1_{G}) = f(1_{G}1_{G}) = f(1_{G})f(1_{G}) \Longrightarrow 1_{G'} = f(1_{G}).$$
\item $f(a^{-1}) = (f(a))^{-1}$ para cada $a \in G$, puesto que $$f(a)f(a^{-1}) = f(aa^{-1}) = f(1_{G}) = 1_{G'},$$ $$f(a^{-1})f(a) = f(a^{-1}a) = f(1_{G}) = 1_{G'}.$$
\item Se llama \textbf{núcleo de $f$} a $$ker f = \lbrace a \in G: f(a) = 1_{G'} \rbrace.$$
El núcleo es un subgrupo normal de $G$. En efecto, dados $a,b \in ker f$ $$f(ab^{-1}) = f(a)f(b^{-1}) = f(a)f(b)^{-1} = 1_{G'}(1_{G'})^{-1} = 1_{G'},$$ probando así que $ker f$ es subgrupo de $G$. Para probar que es normal es suficiente comprobar que $(ker f)^a \subseteq ker f$ para cada $a \in G$. Si $x \in (ker f)^a$ resulta que $axa^{-1} \in ker f$, es decir, $f(axa^{-1}) = 1_{G'}$, o lo que es lo mismo, $f(a)f(x)f(a)^{-1} = 1_{G'}$. Entonces, $f(a)f(x) = 1_{G'}f(a) = f(a) = f(a)1_{G'},$ y simplificando $f(x) = 1_{G'}$, luego $x \in ker f$.
\item $f$ es inyectiva si y sólo si $ker f = \lbrace 1_G \rbrace.$

Supongamos que $f$ es inyectiva. Entonces cada $a \in G$ distinto de $1_G$ cumplirá $$f(a) \neq f(1_G) = 1_{G'},~luego~a \notin ker f.$$ Como $1_G \in ker f$ tendremos que $ker f = \lbrace 1_G \rbrace.$

Recíprocamente, si $ker f = \lbrace 1_G \rbrace$, dados elementos distintos $a,b \in G$, tendremos $ab^{-1} \neq 1_{G'}$, luego $ab^{-1} \notin ker f$, es decir, $$f(ab^{-1}) \neq 1_{G'}.$$ Entonces, $f(a)f(b)^{-1} \neq 1_{G'}$, de donde $f(a) \neq f(b)$ y así $f$ es inyectiva.
\item Se llama \textbf{imagen} de $f$ a la imagen conjuntista, esto es, $$im f = \lbrace f(x) : x \in G \rbrace.$$ La imagen es un subgrupo de $G'$ pues dados $a = f(x), b = f(y)$, $a,b \in im f$, se tiene $$ab^{-1} = f(x)f(y)^{-1} = f(x)f(y^{-1}) = f(xy^{-1})  \in im f.$$
\item Si $f \colon G \longrightarrow G'$ es un homomorfismo de grupos y $H'$ es un subgrupo de $G'$, $$f^{-1}(H') = \lbrace x \in G : f(x) \in H'\rbrace$$ es un subgrupo de $G$. Además si $H$ es subgrupo normal de $G'$, $f^{-1}(H')$ lo es de $G$.

En efecto, si $x,y \in f^{-1}(H')$, entonces $f(x),f(y) \in H'$, de donde $f(xy^{-1}) = f(x)f(y)^{-1} \in H'$, luego $xy^{-1} \in f^{-1}(H')$. Para probar la normalidad de $f^{-1}(H')$ usamos la condición $(3)$ de~\ref{prop:gruNo}: Si $ab \in f^{-1}(H')$ se sigue que $f(a)f(b) = f(ab) \in H'$ y como $H'$ es normal, $f(ba) = f(b)f(a) \in H'.$ Por lo tanto $ba \in f^{-1}(H').$

Observemos que $ker f = f^{-1}(\lbrace 1_{G'} \rbrace )$.

\item Si $H$ es un subgrupo de un grupo $G$, la inclusión $$\begin{array}{rccl}
i\colon &H& \longrightarrow &G\\
&x& \longmapsto &x
\end{array}
$$ es un homomorfismo inyectivo puesto que $i(xy)= xy = i(x)i(y)$ y $x \in ker\hspace{0.1cm} i$ equivale a $i(x) = 1_G$, es decir, $x = 1_G = 1_H.$
\item Si $H$ es un subgrupo normal de un grupo $G$, la proyección  $$\begin{array}{rccl}
\pi\colon &G& \longrightarrow &G/H\\
&x& \longmapsto &xH
\end{array}
$$ es un homomorfismo sobreyectivo. La sobreyectividad es obvia. Además, $\pi (xy) = xyH = (xH)(yH)= \pi(x) \pi(y)$, luego $\pi$ es homomorfismo.
\item Si $f\colon G \longrightarrow  G'$, $g\colon G' \longrightarrow  G''$ son homomorfismos entre grupos, también lo es $g\circ f\colon G \longrightarrow G''$, ya que
$$ (g \circ f)(xy) = g(f(xy)) = g(f(x)f(y)) = g(f(x))g(f(y)) = (g \circ f)(x) (g \circ f)(y).$$
\item Sean $f\colon G \longrightarrow G'$ un homomorfismo de grupos y $x\in G$ un elemento de orden $m$.
\begin{enumerate}
\item $o(f(x))$ divide a $m$.
\item Si $f$ es inyectiva, $o(f(x)) = m$.
\end{enumerate} 

Para lo primero, como $x^{m} = 1$ se tiene que $1 = f(1) = f(x^{m}) = f(x)^{m}$ y así $o(f(x))$ divide a $m$. 

Para lo segundo, sea $k = o(f(x))$. Entonces $f(x)^k = 1$, luego $f(x^k) = 1$ y $x^k \in ker f = \lbrace 1 \rbrace$. Así, $x^k = 1$. De nuevo, $k$ ha de ser múltiplo de $m$. Esto junto con lo visto antes nos da que $k=m$.

\item Supongamos que $S = \lbrace x_1, \ldots, x_p \rbrace$ es un sistema generador de un grupo $G$. Entonces, si $f \colon G \longrightarrow G'$, $g \colon G \longrightarrow G'$ son dos homomorfismos tales que $f(x_i) = g(x_i)$, con $1 \leq i \leq p$, se cumple $f=g$. (Abreviadamente diremos que $f$ queda determinado por las imágenes de $x_1, \ldots, x_p$.)

En efecto: dado $x\in G$ será $x = x_1^{h_1} \ldots x_p^{h_p}, h_1, \ldots, h_p \in \mathbb{Z}$ y así $$f(x) =  f(x_1)^{h_1} \ldots f(x_p)^{h_p} =  g(x_1)^{h_1} \ldots g(x_p)^{h_p} = g(x).$$
\end{enumerate}
\end{observation}

\begin{proposition}[\textbf{\textit{Factorización canónica de un homomorfismo}}]
Sea $f \colon G \longrightarrow G'$ un homomorfismo. Entonces existe un homomorfismo biyectivo $$\bar{f} \colon G/ker f \longrightarrow im f$$ que hace conmutativo el siguiente diagrama $$\xymatrix @=2cm {G\ar[r]^{f} \ar[d]_\pi & G'  \\ G/Ker f \ar[r]^{\bar{f}} & Im f \ar[u]_{i}}$$ donde $i$ y $\pi$ tienen el significado visto en las observaciones $7$ y $8$ anteriores, notemos que $ker f$ es subgrupo normal, y la conmutatividad del diagrama significa que $$f = i \circ \bar{f} \circ \pi.$$
\end{proposition}
\begin{proof}
La última condición nos dice cómo debe ser $\bar{f}$, pues para cada $x \in G$ debe cumplirse:
$$f(x) = (i \circ \bar{f} \circ \pi) (x) =i((\bar{f} \circ \pi)(x)) =i(\bar{f}(\pi(x))) = \bar{f}(\pi(x)) = \bar{f}(x ker f).$$ Y con esto definimos $\bar{f}$. Comprobamos que se cumple lo del enunciado: 
\begin{enumerate}
\item $\bar{f}$ está bien definida ya que si $xker f = yker f$, entonces $$x^{-1}y \in ker f \Longrightarrow f(x^{-1}y) = f(x)^{-1}f(y)=1_{G'}$$ y así $f(x) = f(y) \Longrightarrow \bar{f}(xker f) = \bar{f}(yker f)$.
\item $\bar{f}$ es homomorfismo ya que $$\bar{f}((xkerf)(ykerf)) = \bar{f}(xykerf) = f(xy) = f(x)f(y) = \bar{f}(xkerf)\bar{f}(ykerf).$$
\item $\bar{f}$ es inyectiva ya que si $xkerf \in ker\bar{f}$ entonces 

$f(x) = \bar{f}(xkerf) = 1_{im f} = 1_{G'} \Longrightarrow x \in kerf$ y así $xkerf = kerf$, que es el elemento neutro de $G/ker f$ y así $\bar{f}$ es inyectiva.
\item $\bar{f}$ es sobreyectiva, pues cada elemento de $im f$ es de la forma $im f\ni g = f(x) = \bar{f}(xkerf)$ para cierto $x \in G.$

\end{enumerate}

Además la conmutatividad del diagrama es obvia, pues $\bar{f}$ se ha definido para que cumpla esta condición.

\end{proof}

\begin{definition}Un homomorfismo biyectivo entre dos grupos se llama \textbf{isomorfismo}. Cuando exista un isomorfismo $f \colon G \longrightarrow G'$ diremos que los grupos $G$ y $G'$ son \textbf{isomorfos}, y escribiremos $G \simeq G'$.
\end{definition}

\begin{observation}Algunas observaciones:
\begin{enumerate}
\item El término «$G$ y $G'$ son isomorfos» no es ambiguo, pues si $f \colon G \longrightarrow G'$ es isomorfismo, también lo es $f^{-1}\colon G' \longrightarrow G$. 

En efecto, como la inversa de toda aplicación biyectiva es biyectiva también, bastará comprobar que $f^{-1}$ es homomorfismo de grupos.

Ahora bien, si $a,b \in G'$ y $f^{-1}(a) = x$, $f^{-1}(b) = y$, se tiene $f(x) = a, f(y) = b$, luego $f(xy) = f(x) f(y) = ab,$ es decir, $xy = f^{-1}(ab)$, de donde $f^{-1}(ab) = f^{-1}(a)f^{-1}(b)$.
\item Dos grupos isomorfos tienen las «mismas propiedades» (siempre que sean propiedades de la teoría de grupos). Por ejemplo 
\end{enumerate}
\end{observation}

\subsection{Teoremas de Isomorfía}
\subsection{Teorema de estructura de los grupos abelianos finitos}
\section{Grupos abelianos finitamente generados. Acciones de grupos sobre conjuntos}
\subsection{Grupos abelianos finitamente generados}
\subsection{Algoritmo para la obtención del número de Betti y los coeficientes de torsión}
\subsection{Generadores y relaciones}
\subsection{Acciones de grupos sobre conjuntos}
\end{document}